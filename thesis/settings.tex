%! Author = clem
%! Date = 23.11.20
\documentclass[
			twoside,
            fontsize=12pt,
            parskip=half-,
            %DIV=calc,
            titlepage,
            abstract=true, % prints Abstractname
            %listof=totocnumbered, % LOF, LOT, LOL  nummeriert im TOC
            bibliography=totocnumbered, % Referenzen nummeriert im TOC
            headings=normal,% Überschriften verkleinern
          	]{scrreprt}% scrreprt

%%%%%%%%%%%%%%%% Pakete %%%%%%%%%%%%%%%%
\usepackage{blindtext}

\usepackage{geometry}
\usepackage{calc}
\usepackage{amsmath}	% MUSS vor fontspec geladen werden
\usepackage{mathtools}	% modifiziert amsmath
\usepackage{amssymb}	% mathematische symbole, für \ceckmarks
\usepackage{algorithm,algpseudocode}
\usepackage{scrlayer}
\usepackage{fontspec}
\usepackage{beramono}

\usepackage{microtype}
\usepackage[english]{babel}
\usepackage{lmodern}
%\usepackage{slantsc}
\usepackage[scale]{tgheros} % kapitälchen mit helvetica (ähnlich der arial)

\usepackage{scrhack} % ersetzt float
\usepackage{enumitem}
\usepackage{etoolbox}
\usepackage{multicol}
\usepackage{listings}
\usepackage{graphicx}
\usepackage{tabularx}
\usepackage{booktabs} % To thicken table lines
\usepackage[RGB,table]{xcolor}
\usepackage{multirow}        % tabular-multirow
\usepackage{colortbl}
\usepackage{csquotes}		 % inline quotation
\usepackage{comment} 		 % comment large blocks
\usepackage[colorlinks,      % hyperlinks
            linktocpage,
            allcolors=FUblue]
            {hyperref}
\usepackage{titling}
% Kopf- und Fußzeile:
\usepackage[headsepline=true,
            footsepline=true, % plainfootsepline,
            markcase=upper,
            manualmark, % manualmark / automark,
            headsepline=0.5pt, footsepline=0.5pt
            ]{scrlayer-scrpage}
%\usepackage{lastpage} % total page number

%\usepackage{biolinum} % notwendig für small capitals (läuft nicht)
%\usepackage{tocloft} % für cfttoctitlefont, um Schriftgröße zu verringern


%%%%%%%%%%%%%% Settings %%%%%%%%%%%%%%
\setkomafont{sectioning}{\rmfamily\bfseries}% setzt Ü-Schriften in Serifen, {disposition}
\rowcolors{2}{lightgray!20}{lightgray!40} % alternierende Zeilenfarbe
\setcounter{secnumdepth}{3}
\setcounter{tocdepth}{2}
%\setcounter{lofdepth}{1}
%\setcounter{lotdepth}{2}

%%%%%%%%%%%%%% Layout %%%%%%%%%%%%%%
\setlength{\unitlength}{1cm}
\newcommand{\abstand}{\dimexpr\paperwidth/18\relax}
\newcommand{\rand}{\dimexpr3.6\abstand\relax}
\newcommand{\randI}{\dimexpr1.8\abstand\relax}
\newcommand{\randA}{\dimexpr1.8\abstand\relax}
\newcommand{\logosize}{\dimexpr1.5cm\relax}
\setlength{\columnsep}{\abstand}
\ModifyLayer[addvoffset=-3pt]{scrheadings.foot.above.line}% Linie über Fußzeile nach oben schieben
%\ModifyLayer[addvoffset=+3pt]{scrheadings.head.below.line}% Linie unter Kopfzeile nach unten schieben
\setlength{\footheight}{\dimexpr2\abstand+15ex\relax}
\geometry{
    headheight = 1cm,
    headsep = \abstand,
    footskip = \dimexpr\abstand+\footheight\relax,
	left = \randI,
    right = \randA,
	top = \dimexpr1.8\abstand+\headheight+\headsep\relax,
    bottom = \dimexpr\abstand+\footskip\relax,
    marginparwidth = 0pt,
    bindingoffset = \abstand,
    %showframe,
    %includehead, includefoot
}

%%%%%%%%%%%%%% eigene Farben %%%%%%%%%%%%%%
\let\definecolor=\xdefinecolor
\definecolor{FUgreen}{RGB}{153,204,0}
\definecolor{FUblue}{RGB}{0,51,102}
\definecolor{DBblue}{RGB}{0,102,204}
\definecolor{DBbluedark}{RGB}{0,51,102}
\definecolor{GlobalBlack}{RGB}{0,0,0}

% setzt die globale Schriftfarbe auf DBbluedark
\makeatletter
\newcommand{\globalcolor}[1]{%
  \color{#1}\global\let\default@color\current@color
}
\makeatother
\AtBeginDocument{\globalcolor{GlobalBlack}}

% färbt die Chapter- und Sectionnummern ein
\renewcommand*{\chapterformat}{%
  \textcolor{DBblue}{\thechapter}\autodot\enskip%
}
\renewcommand*{\sectionformat}{%
  \textcolor{DBblue}{\thesection}\autodot\enskip%
}
\renewcommand*{\subsectionformat}{%
  \textcolor{DBblue}{\thesubsection}\autodot\enskip%
}
\renewcommand*{\subsubsectionformat}{%
  \textcolor{DBblue}{\thesubsubsection}\autodot\enskip%
}

%%%%%%%%%%%% Kopf- und Fußzeile:
\clearpairofpagestyles% Bisherige Einstellungen löschen
%\automark[section]{chapter}
\renewcommand*\chapterpagestyle{scrheadings} % scrheadings, auch Seiten bei Chapterbeginn erhalten Kopf- und Fußzeilen
% link: http://texwelt.de/wissen/fragen/6302/wo-sind-kopfzeile-und-fuzeile-bei-kapitelstart

\renewcommand{\chaptermark}[1]{\markboth{#1}{#1}} % \markright, \markleft, \markboth

\pagestyle{scrheadings}

%\lehead*{}
%\cehead*{}
%\rehead*{\upshape\thechapter, \headmark} % \thechapter, \headmark [\headmark]
%%\lefoot*{}%\pageref{LastPage}
%\cefoot*{}
%\refoot*{}%\pagemark}%\parbox{\dimexpr\linewidth-2em\relax}{\upshape \textbf{\theauthor}, \textit{\thetitle}, \Year}}

%\lohead*{}%\upshape\thechapter\xspace \headmark}
%\cohead*{}
%\rohead*{}
%\lofoot*{}%\pagemark}%\parbox{\dimexpr\linewidth-2em\relax}{\upshape \textbf{\theauthor}, \textit{\thetitle}, \Year}}
%%\cofoot*{}%center odd foot}
%\rofoot*{}%\pagemark}%\pageref{LastPage}
%\pagestyle{scrheadings}

%\KOMAoptions{onpsinit=\linespread{3}\selectfont}% Kopf- und Fußzeilen dreizeilig
%\renewcommand*\chaptermarkformat{\thechapter\autodot\enskip}% Kopfzeile ohne chapterprefix

%%%%%%%%%%%%%% Fonts %%%%%%%%%%%%%%
\setmainfont{Roboto}
\setsansfont{TeX Gyre Heros}
\setmonofont{Bitstream Vera Sans Mono}
\renewcommand\UrlFont{\rmfamily} % passt URL an

% definiert textsc neu, damit es mit Arial funktioniert
\renewcommand\textsc[1]{{\sffamily\scshape#1}}

%%%%%%%%%%%%%%% Fontsize %%%%%%%%%%%%%%
% \setkomafont{chapter}{\LARGE}
% \setkomafont{section}{\Large}
% \setkomafont{subsection}{\large}
% \setkomafont{paragraph}{\normalsize}
%\renewcommand\cfttoctitlefont{\LARGE\bfseries}

%%%%%%%%%% Referenzen %%%%%%%%%%%
%\usepackage[style=numeric, bibencoding=utf8, backend=biber, sorting=nty, maxbibnames=15]{biblatex}% für bibliographie, style=authortitle/numeric/ieee, backend=biber/bibtex/bibtex8, bibencoding=ascii/utf8
\usepackage{silence}% Filter warnings issued by package biblatex starting with "Patching footnotes failed"
\WarningFilter{biblatex}{Patching footnotes failed}

%\bibliographystyle{unsrt}	 % reorders the cited references by apperance
%\bibliographystyle{plain}
%\addbibresource{content/references.bib}
%\DeclareFieldFormat{isbn}{}
%\renewcommand{\bibname}{\thesection References}

\usepackage{xpatch}
\xpatchbibmacro{author}{\printnames{author}}{\mkbibbold{\printnames{author}}}{}{} % author bold

%%%%%%%%%%%%%% Enumitem
\setlist[enumerate]{font=\color{DBblue}\bfseries, leftmargin=*}
\setlist[enumerate,1]{label=\arabic*, ref=\alph*.}
\setlist[enumerate,2]{label*=.\arabic*, ref=\alph*}
\setlist[enumerate,3]{label*=.\arabic*, ref=\alph*}
\setlist[itemize]{font=\color{DBblue}\bfseries, leftmargin=*}
\setlist[itemize,1]{label=$\blacktriangleright$, ref=\labelitemi}
\setlist[itemize,2]{label=\small$\blacktriangleright$, ref=\labelitemii}
\setlist[itemize,3]{label=\footenotesize$\blacktriangleright$, ref=\labelitemiii}%textbullet, diamond, bullet
\setlist[description]{font=\color{DBblue}\bfseries, leftmargin=*}

%%%%%%%%%%%%%%% Tables %%%%%%%%%%%%%%
\renewcommand{\arraystretch}{1.2}
\aboverulesep=0ex
\belowrulesep=0ex

%%%%%%%%%%%%%% Chapter %%%%%%%%%%%%
% reduces the space before chapter to zero
\renewcommand*{\chapterheadstartvskip}{\vspace*{\dimexpr5pt-\topskip\relax}}

%\RedeclareSectionCommand[beforeskip=\baselineskip, afterskip=\baselineskip]{chapter}
%\RedeclareSectionCommand[beforeskip=0.75\baselineskip, afterskip=0.5\baselineskip]{section}
%\RedeclareSectionCommand[beforeskip=0.75\baselineskip, afterskip=0.5\baselineskip]{subsection}
\RedeclareSectionCommand[beforeskip=0.75\baselineskip, afterskip=0pt]{paragraph}
\let\paragraphOld\paragraph
\renewcommand{\paragraph}[1]{\paragraphOld{#1}\hspace{1ex}}

%%%%%%%%%%%%%% Syntaxhighlighting %%%%%%%%%%%%%%
\renewcommand{\lstlistingname}{Source Code}% Listing -> Source Code
\lstloadlanguages{Python, C, C++, R, HTML, Haskell, Java, SQL, PHP, SPARQL, Octave, Matlab, OCL, XML, bash, Ruby}
\lstset{
   basicstyle=\color{DBbluedark}\small\ttfamily\selectfont,	% \scriptsize the size of the fonts that are used for the code
   backgroundcolor = \color{gray!20},	% legt Farbe der Box fest
   breakatwhitespace=false,	% sets if automatic breaks should only happen at whitespace
   breaklines=true,			% sets automatic line breaking
   captionpos=b,				% sets the caption-position to bottom, t for top
   commentstyle=\color{gray}\selectfont,% comment style
   frame=single,				% adds a frame around the code
   keepspaces=true,			% keeps spaces in text, useful for keeping indentation
							% of code (possibly needs columns=flexible)
   keywordstyle=\bfseries\color{DBblue}\selectfont,% keyword style
   ndkeywordstyle=\color{DBbluedark}\bfseries\selectfont,
   numbers=left,				% where to put the line-numbers;
   							% possible values are (none, left, right)
   numberstyle=\scriptsize\color{FUblue}\selectfont,	% the style that is used for the line-numbers
   numbersep=9pt,			% how far the line-numbers are from the code
   stepnumber=1,				% nummeriert nur jede i-te Zeile
   showspaces=false,			% show spaces everywhere adding particular underscores;
							% it overrides 'showstringspaces'
   %showstringspaces=false,	% underline spaces within strings only
   showtabs=false,			% show tabs within strings adding particular underscores
   flexiblecolumns=false,
   %tabsize=1,				% the step between two line-numbers. If 1: each line will be numbered
   stringstyle=\color{red}\ttfamily\selectfont,	% string literal style
   numberblanklines=false,				% leere Zeilen werden nicht mitnummeriert
   xleftmargin=\dimexpr2em\relax,	% Abstand zum linken Layoutrand
   framexleftmargin=\dimexpr2em-3pt\relax,
   xrightmargin=\dimexpr2pt\relax,					% Abstand zum rechten Layoutrand
   framexrightmargin=\dimexpr-2pt\relax,
   aboveskip=2ex,
}

%%%%%%%%%%%%%%%% definition of language %%%%%%%%%%%%%%%%
\lstdefinestyle{html}{
   language=HTML,
}
\lstdefinestyle{c}{
  language=C,
}
\lstdefinestyle{cpp}{
  language=C++,
}
\lstdefinestyle{py}{
   language=Python,
}
\lstdefinestyle{php}{
  language=PHP,
}
\lstdefinestyle{rdf}{%oder xml
  language=SPARQL,
}
\lstdefinestyle{o}{
  language=Octave,
}
\lstdefinestyle{m}{
  language=Matlab,
}
\lstdefinestyle{ocl}{%omg
  language=OCL,
}
\lstdefinestyle{xml}{
  language=XML,
}
\lstdefinestyle{hs}{
  language=Haskell,
}
\lstdefinestyle{sql}{
  language=SQL,
}
\lstdefinestyle{bash}{
  language=bash,
}
\lstdefinestyle{pseudo}{
  language = Python,
  mathescape = true,
  morekeywords = {do, procedure, end, while, if, else},
}
\lstdefinestyle{ruby}{
  language=Ruby,
}
\lstdefinestyle{java}{
	language=Java,
    morekeywords={typeof, new, true, false, function, return, null, catch, switch, var, if, in, while, do, else, case, break, class, export, boolean, throw, implements, extends, import, this, int, Integer, Boolean, String, Object, Float, Double, double, float, long, void},
    ndkeywords={List, E},
	extendedchars=true,% lets you use non-ASCII characters;
    % for 8-bits encodings only, does not work with UTF-8
}
\lstdefinestyle{R}{
	language=R,
	extendedchars=true,% for 8-bits encodings only,
    % does not work with UTF-8
}

% environments: abstract
\newenvironment{abstractEN}{%
\thispagestyle{empty}%
\selectlanguage{english}%
\begin{center}%
\textbf{\abstractname}
\end{center}}{}

\newenvironment{abstractDE}{%
\thispagestyle{empty}%
%\selectlanguage{ngerman}%
\begin{center}%
\textbf{Abstract (german)}
\end{center}}{}
