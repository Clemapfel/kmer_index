%%% MOTIVATION %%%
\chapter{Introduction}
A k-mer (also known as "q-gram", henceforth written as "kmer") is any oligonucleotide of length $k$ usually in the
context of a sub-sequence of a longer genome. A kmer-index is a data structure that for every unique kmer
stores the position of all occurences in a given text, usually relative to the beginning of the text.
This allows the user to efficiently look up both how often and exactly where a kmer occurs.
Analysis of kmers is used in de novo (meta-)genome\cite{megahit} and transcriptome\cite{SOAPdenovo-Trans} assembly.
In these applications efficiently accessing short reads from a set of next-generation sequencing data as well as sorting
kmer that represent the edges of for example de bruijin graphs during assembly may present a
"[performance] bottleneck" (Li et. al) that may be somewhat alleviated with an efficient implementation of a kmer index.
\newline
A kmer spectrum is a frequency spectrum that for a given k depicts how many occurences kmer of a certain multiplicity in a text
exhibit. %how many of the kmer that have x copies in the text are there
Because the kmer index provides the number of occurences of each kmer, building the kmer-index inherently also yields
the kmer spectrum. Kmer spectra have been used along traditional biomarkers for phenotypic classification of cancer
metagenome samples via the aid of machine learning\cite{phenotype:classification:with:kmer:spectrum}, pairwise dissimilarity
analysis comparing metagenomes of arbitrary origin\cite{kmer:spectrum:dissimilarity} as well as assigning taxonomic labels
to such metagenomes\cite{kraken:metagenome:classification}. Kmer spectra have furthermore been utilized for estimating genome
size without the need of assembly\cite{genome:size:estimation} and for substitution-based error correction
of next generation illumina sequencing data\cite{musket:kmer:spectrum:error:correction}.
\newline
Given the wide applicability of efficiently searching kmer a wide variety of tools and libraries have been developed.
One such tool is \href{https://github.com/seqan/seqan3}{Seqan3}s fm\_index \cite{fm:index:master:thesis}, an implementation
of the commonly used fm-index first created by P. Ferragina and G. Manzini \cite{original:fm} that aims to provide a generalist
tool to finding sequences both exact and non-exact in arbitrary genomic data.
Unlike the fm-index the kmer-index implementation does not allow for searching of non-exact sequences to allow for further
optimization and is thus more deliberatly aimed at specifcially searching kmer. This paper aims to compare the applicability
of the kmer-index implementation with the Seqan3s fm\_index for this purpose and attempt to illuminate wether utilizing
the kmer-indices inherent strengths may results in an overall performance boost.