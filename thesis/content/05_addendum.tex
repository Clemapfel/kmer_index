
\chapter{Addendum}
\section{\label{Addendum: Correctness}Assuring Correctness}

To assure that the kmer index returns the correct positions a \href{https://github.com/google/googletest}{test function}
was written that repeatedly compares search results of the kmer-index and
the seqan3s fm\_index for randomized queries and texts. This asserts that the
fm\_index itself is bug-free and as both indices are meant to be used
in the same library this assertion was presumed to be reasonable.

\begin{algorithm}[H]
\begin{verse}
\textbf{Input}: seed, $n$, $m$

\textbf{while} $(n>0)$ \textbf{do}
\begin{verse}
\textbf{let} text $\leftarrow$ generate\_sequence($m$, seed)

\textbf{let} fm $\leftarrow$ kmer\_index(text)

\textbf{let} kmer $\leftarrow$ fm\_index(text)

\textbf{for }$q$ \textbf{in} \{k-1, k, k+1, ..., 2{*}k\} \textbf{do}
\begin{verse}
\textbf{let} query $\leftarrow$ generate\_sequence(q, seed)

\textbf{let} result $_{fm}$$\leftarrow$ fm.search(query)

\textbf{let} result $_{kmer}$$\leftarrow$ kmer.search(query)

\textbf{assert} (result$_{fm}$ = result$_{kmer}$)
\end{verse}

\textbf{end}

seed $\leftarrow$ seed + 1
n $\leftarrow$ n-1
\end{verse}

\textbf{end}
\end{verse}
\caption{Test function comparing kmer- and fm-index results to assure correctness.}
\end{algorithm}

Given enough time and iteration over a big enough text it is reasonable to assume
that this function will uncover possible bugs by itself. The text was furthermore manually
modified to account for edge cases (such
as the query happening at the very end of the text as detailed in
Section\ref{section m < k}). If a discrepancy between the fm- and kmer-index was uncovered
the function would report the exact seed and query size. As all the randomness in the above
algorithm is determinstic this made it easy to reproduce possible bugs. Before each benchmark run
whos results where presented in this paper correcteness was assured with $n>10{{}^6}$ iterations
using text sizes as large as the machines memory allowed for. During these runs no bugs were found.

\section{Performance Optimization}
Implementation of the kmer-index was guided at every step by benchmarking
newly implemented components and comparing their performance against
other possible implementations. In this section the most relevant
of these decisions are explained. Note that C++20 was used for all
performance relevant code.

\subsection{Choosing the fastest Pow Implementation}
\subsubsection{Methods}
By the nature of the hash, exponentiation (henceforth referred to
as "pow" ()in reference to the commonly used \lstinline{std::pow})
is used every search call, sometimes multiple times. Four different versions of pow were
implemented and their performance evaluated. Note that for the purpose of the kmer-index the pow implementation
will only ever be used with positive integers and has to be able to be evaluated at compile time.

\begin{itemize}
\item \lstinline{trivial_pow(base,n)}: A trivial implementation calling $base*base$ $n$-many
times
\item \lstinline{recursive_pow(base,n)}: Utilizing a recursive approach,
this function calls itself recursively $n$-many times and then evaluates
each call from the inside out to return the correct result based on
whether the exponent was odd or even
\item \lstinline{bit_pow(base,n)}: Utilizes bit-operations which are generally
more well-optimized on most modern machines
\item \lstinline{switch_pow(base,n)}: Instead of using a loop, this implementation
has multiple switch cases with identical code, when the function is
called a lookup in a pre-calculated table produces the correct first
switch case to start with. The result then "falls through" the rest
of the switch cases the correct number of times. Any exponentiation
that would overflow the unsigned 64 bit integer result is immediately
caught and 0 is returned instead.
\end{itemize}
\subsubsection{Results}
\begin{figure}[H]
\includegraphics[width=1\textwidth]{/home/clem/Workspace/kmer_index/source/benchmarks/pow_vs_pow/pow_vs_pow}

\caption{Boxplot showing runtime distribution of average time to compute $x^{y}$.}
\end{figure}

Results indicate that the implementation utilizing the fall-through
switch is overall the fastest and was thus used whenever
possible.

\subsection{Paralellization}
Other than the amount of memory needed the only true disadvantage of the
multi kmer-index is the fact that the time it takes to construct it increases drastically with the
number of $k_{i}$ specified. To address this a general purpose thread pool was implemented that
allows all of the single kmer-index elements to be constructed in parallel.
\begin{minipage}{\linewidth}
\begin{lstlisting}[caption={Paralell invocation of the create function for individual kmer-index
elements during construction of the multi kmer-index.}]
template<std::ranges::range text_t>
multi_kmer_index(text_t& text)
	: single_kmer_index<ks>()...
{
	auto pool = kmer::detail::thread_pool{(...)};
	std::vector<std::future<(...)>> futures;
	(futures.emplace_back(
		pool.execute(&single_kmer_index<ks>::create, text)), ...);

	// wait to finish
	for (auto& f : futures)
		f.get();
}
\end{lstlisting}
\end{minipage}

As the number of possible $k$ is currently restriced to at most 31, modern systems with 32 or more CPU cores are
capable of constructing a single multi kmer-index with optimal coverage with no additional runtime overhead compared
to a single kmer-index.