%! Author = clem
%! Date = 23.11.20

%! Author = clem
%! Date = 23.11.20
\documentclass[
			twoside,
            fontsize=12pt,
            parskip=half-,
            %DIV=calc,
            titlepage,
            abstract=true, % prints Abstractname
            %listof=totocnumbered, % LOF, LOT, LOL  nummeriert im TOC
            bibliography=totocnumbered, % Referenzen nummeriert im TOC
            headings=normal,% Überschriften verkleinern
          	]{scrreprt}% scrreprt

%%%%%%%%%%%%%%%% Pakete %%%%%%%%%%%%%%%%
\usepackage{blindtext}

\usepackage{geometry}
\usepackage{calc}
\usepackage{amsmath}	% MUSS vor fontspec geladen werden
\usepackage{mathtools}	% modifiziert amsmath
\usepackage{amssymb}	% mathematische symbole, für \ceckmarks
\usepackage{algorithm,algpseudocode}
\usepackage{fancyhdr}
\usepackage{fontspec}
\usepackage{beramono}

\usepackage{microtype}
\usepackage[english]{babel}
\usepackage{lmodern}
%\usepackage{slantsc}
\usepackage[scale]{tgheros} % kapitälchen mit helvetica (ähnlich der arial)

\usepackage{scrhack} % ersetzt float
\usepackage{enumitem}
\usepackage{etoolbox}
\usepackage{multicol}
\usepackage{listings}
\usepackage{graphicx}
\usepackage{tabularx}
\usepackage{booktabs} % To thicken table lines
\usepackage[RGB,table]{xcolor}
\usepackage{multirow}        % tabular-multirow
\usepackage{colortbl}
\usepackage{csquotes}		 % inline quotation
\usepackage{comment} 		 % comment large blocks
\usepackage[colorlinks,      % hyperlinks
            linktocpage,
            allcolors=FUblue]
            {hyperref}
\usepackage{titling}
% Kopf- und Fußzeile:
\usepackage[headsepline=true,
            footsepline=true, % plainfootsepline,
            markcase=upper,
            manualmark, % manualmark / automark,
            headsepline=0.5pt, footsepline=0.5pt
            ]{scrlayer-scrpage}
%\usepackage{lastpage} % total page number

%\usepackage{biolinum} % notwendig für small capitals (läuft nicht)
%\usepackage{tocloft} % für cfttoctitlefont, um Schriftgröße zu verringern


%%%%%%%%%%%%%% Settings %%%%%%%%%%%%%%
\setkomafont{sectioning}{\rmfamily\bfseries}% setzt Ü-Schriften in Serifen, {disposition}
\rowcolors{2}{lightgray!20}{lightgray!40} % alternierende Zeilenfarbe
\setcounter{secnumdepth}{3}
\setcounter{tocdepth}{2}
%\setcounter{lofdepth}{1}
%\setcounter{lotdepth}{2}

%%%%%%%%%%%%%% Layout %%%%%%%%%%%%%%
\setlength{\unitlength}{1cm}
\newcommand{\abstand}{\dimexpr\paperwidth/18\relax}
\newcommand{\rand}{\dimexpr3.6\abstand\relax}
\newcommand{\randI}{\dimexpr1.8\abstand\relax}
\newcommand{\randA}{\dimexpr1.8\abstand\relax}
\newcommand{\logosize}{\dimexpr1.5cm\relax}
\setlength{\columnsep}{\abstand}
\ModifyLayer[addvoffset=-3pt]{scrheadings.foot.above.line}% Linie über Fußzeile nach oben schieben
%\ModifyLayer[addvoffset=+3pt]{scrheadings.head.below.line}% Linie unter Kopfzeile nach unten schieben
\setlength{\footheight}{\dimexpr2\abstand+15ex\relax}
\geometry{
    headheight = 1cm,
    headsep = \abstand,
    footskip = \dimexpr\abstand+\footheight\relax,
	left = \randI,
    right = \randA,
	top = \dimexpr1.8\abstand+\headheight+\headsep\relax,
    bottom = \dimexpr\abstand+\footskip\relax,
    marginparwidth = 0pt,
    bindingoffset = \abstand,
    %showframe,
    %includehead, includefoot
}

%%%%%%%%%%%%%% eigene Farben %%%%%%%%%%%%%%
\let\definecolor=\xdefinecolor
\definecolor{FUgreen}{RGB}{153,204,0}
\definecolor{FUblue}{RGB}{0,51,102}
\definecolor{DBblue}{RGB}{0,102,204}
\definecolor{DBbluedark}{RGB}{0,51,102}
\definecolor{GlobalBlack}{RGB}{0,0,0}

% setzt die globale Schriftfarbe auf DBbluedark
\makeatletter
\newcommand{\globalcolor}[1]{%
  \color{#1}\global\let\default@color\current@color
}
\makeatother
\AtBeginDocument{\globalcolor{GlobalBlack}}

% färbt die Chapter- und Sectionnummern ein
\renewcommand*{\chapterformat}{%
  \textcolor{DBblue}{\thechapter}\autodot\enskip%
}
\renewcommand*{\sectionformat}{%
  \textcolor{DBblue}{\thesection}\autodot\enskip%
}
\renewcommand*{\subsectionformat}{%
  \textcolor{DBblue}{\thesubsection}\autodot\enskip%
}
\renewcommand*{\subsubsectionformat}{%
  \textcolor{DBblue}{\thesubsubsection}\autodot\enskip%
}

%%%%%%%%%%%% Kopf- und Fußzeile:
\clearpairofpagestyles% Bisherige Einstellungen löschen
%\automark[section]{chapter}
\renewcommand*\chapterpagestyle{scrheadings} % scrheadings, auch Seiten bei Chapterbeginn erhalten Kopf- und Fußzeilen
% link: http://texwelt.de/wissen/fragen/6302/wo-sind-kopfzeile-und-fuzeile-bei-kapitelstart

\renewcommand{\chaptermark}[1]{\markboth{#1}{#1}} % \markright, \markleft, \markboth


\pagestyle{scrheadings}

\lehead*{}
\rehead*{Chapter \upshape\thechapter: \headmark}
\cehead*{}
\lefoot*{}
\refoot*{\pagemark}
\cefoot*{};


\lohead*{Chapter \upshape\thechapter: \headmark}
\rohead*{}
\cohead*{}
\lofoot*{\pagemark}
\rofoot*{}
\cofoot*{}

%\lehead*{}
%\cehead*{}
%\rehead*{\upshape\thechapter, \headmark} % \thechapter, \headmark [\headmark]
%%\lefoot*{}%\pageref{LastPage}
%\cefoot*{}
%\refoot*{}%\pagemark}%\parbox{\dimexpr\linewidth-2em\relax}{\upshape \textbf{\theauthor}, \textit{\thetitle}, \Year}}

%\lohead*{}%\upshape\thechapter\xspace \headmark}
%\cohead*{}
%\rohead*{}
%\lofoot*{}%\pagemark}%\parbox{\dimexpr\linewidth-2em\relax}{\upshape \textbf{\theauthor}, \textit{\thetitle}, \Year}}
%%\cofoot*{}%center odd foot}
%\rofoot*{}%\pagemark}%\pageref{LastPage}
%\pagestyle{scrheadings}

%\KOMAoptions{onpsinit=\linespread{3}\selectfont}% Kopf- und Fußzeilen dreizeilig
%\renewcommand*\chaptermarkformat{\thechapter\autodot\enskip}% Kopfzeile ohne chapterprefix

%%%%%%%%%%%%%% Fonts %%%%%%%%%%%%%%
\setmainfont{Roboto}
\setsansfont{TeX Gyre Heros}
\setmonofont{Bitstream Vera Sans Mono}
\renewcommand\UrlFont{\rmfamily} % passt URL an

% definiert textsc neu, damit es mit Arial funktioniert
\renewcommand\textsc[1]{{\sffamily\scshape#1}}

%%%%%%%%%%%%%%% Fontsize %%%%%%%%%%%%%%
% \setkomafont{chapter}{\LARGE}
% \setkomafont{section}{\Large}
% \setkomafont{subsection}{\large}
% \setkomafont{paragraph}{\normalsize}
%\renewcommand\cfttoctitlefont{\LARGE\bfseries}

%%%%%%%%%% Referenzen %%%%%%%%%%%
%\usepackage[style=numeric, bibencoding=utf8, backend=biber, sorting=nty, maxbibnames=15]{biblatex}% für bibliographie, style=authortitle/numeric/ieee, backend=biber/bibtex/bibtex8, bibencoding=ascii/utf8
\usepackage{silence}% Filter warnings issued by package biblatex starting with "Patching footnotes failed"
\WarningFilter{biblatex}{Patching footnotes failed}

%\bibliographystyle{unsrt}	 % reorders the cited references by apperance
%\bibliographystyle{plain}
%\addbibresource{content/references.bib}
%\DeclareFieldFormat{isbn}{}
%\renewcommand{\bibname}{\thesection References}

\usepackage{xpatch}
\xpatchbibmacro{author}{\printnames{author}}{\mkbibbold{\printnames{author}}}{}{} % author bold

%%%%%%%%%%%%%% Enumitem
\setlist[enumerate]{font=\color{DBblue}\bfseries, leftmargin=*}
\setlist[enumerate,1]{label=\arabic*, ref=\alph*.}
\setlist[enumerate,2]{label*=.\arabic*, ref=\alph*}
\setlist[enumerate,3]{label*=.\arabic*, ref=\alph*}
\setlist[itemize]{font=\color{DBblue}\bfseries, leftmargin=*}
\setlist[itemize,1]{label=$\blacktriangleright$, ref=\labelitemi}
\setlist[itemize,2]{label=\small$\blacktriangleright$, ref=\labelitemii}
\setlist[itemize,3]{label=\footenotesize$\blacktriangleright$, ref=\labelitemiii}%textbullet, diamond, bullet
\setlist[description]{font=\color{DBblue}\bfseries, leftmargin=*}

%%%%%%%%%%%%%%% Tables %%%%%%%%%%%%%%
\renewcommand{\arraystretch}{1.2}
\aboverulesep=0ex
\belowrulesep=0ex

%%%%%%%%%%%%%% Chapter %%%%%%%%%%%%
% reduces the space before chapter to zero
\renewcommand*{\chapterheadstartvskip}{\vspace*{\dimexpr5pt-\topskip\relax}}

%\RedeclareSectionCommand[beforeskip=\baselineskip, afterskip=\baselineskip]{chapter}
%\RedeclareSectionCommand[beforeskip=0.75\baselineskip, afterskip=0.5\baselineskip]{section}
%\RedeclareSectionCommand[beforeskip=0.75\baselineskip, afterskip=0.5\baselineskip]{subsection}
\RedeclareSectionCommand[beforeskip=0.75\baselineskip, afterskip=0pt]{paragraph}
\let\paragraphOld\paragraph
\renewcommand{\paragraph}[1]{\paragraphOld{#1}\hspace{1ex}}

%%%%%%%%%%%%%% Syntaxhighlighting %%%%%%%%%%%%%%
\renewcommand{\lstlistingname}{Source Code}% Listing -> Source Code
\lstloadlanguages{Python, C, C++, R, HTML, Haskell, Java, SQL, PHP, SPARQL, Octave, Matlab, OCL, XML, bash, Ruby}
\lstset{
   basicstyle=\color{DBbluedark}\small\ttfamily\selectfont,	% \scriptsize the size of the fonts that are used for the code
   backgroundcolor = \color{gray!20},	% legt Farbe der Box fest
   breakatwhitespace=false,	% sets if automatic breaks should only happen at whitespace
   breaklines=true,			% sets automatic line breaking
   captionpos=b,				% sets the caption-position to bottom, t for top
   commentstyle=\color{gray}\selectfont,% comment style
   frame=single,				% adds a frame around the code
   keepspaces=true,			% keeps spaces in text, useful for keeping indentation
							% of code (possibly needs columns=flexible)
   keywordstyle=\bfseries\color{DBblue}\selectfont,% keyword style
   ndkeywordstyle=\color{DBbluedark}\bfseries\selectfont,
   numbers=left,				% where to put the line-numbers;
   							% possible values are (none, left, right)
   numberstyle=\scriptsize\color{FUblue}\selectfont,	% the style that is used for the line-numbers
   numbersep=9pt,			% how far the line-numbers are from the code
   stepnumber=1,				% nummeriert nur jede i-te Zeile
   showspaces=false,			% show spaces everywhere adding particular underscores;
							% it overrides 'showstringspaces'
   %showstringspaces=false,	% underline spaces within strings only
   showtabs=false,			% show tabs within strings adding particular underscores
   flexiblecolumns=false,
   %tabsize=1,				% the step between two line-numbers. If 1: each line will be numbered
   stringstyle=\color{red}\ttfamily\selectfont,	% string literal style
   numberblanklines=false,				% leere Zeilen werden nicht mitnummeriert
   xleftmargin=\dimexpr2em\relax,	% Abstand zum linken Layoutrand
   framexleftmargin=\dimexpr2em-3pt\relax,
   xrightmargin=\dimexpr2pt\relax,					% Abstand zum rechten Layoutrand
   framexrightmargin=\dimexpr-2pt\relax,
   aboveskip=2ex,
}

%%%%%%%%%%%%%%%% definition of language %%%%%%%%%%%%%%%%
\lstdefinestyle{html}{
   language=HTML,
}
\lstdefinestyle{c}{
  language=C,
}
\lstdefinestyle{cpp}{
  language=C++,
}
\lstdefinestyle{py}{
   language=Python,
}
\lstdefinestyle{php}{
  language=PHP,
}
\lstdefinestyle{rdf}{%oder xml
  language=SPARQL,
}
\lstdefinestyle{o}{
  language=Octave,
}
\lstdefinestyle{m}{
  language=Matlab,
}
\lstdefinestyle{ocl}{%omg
  language=OCL,
}
\lstdefinestyle{xml}{
  language=XML,
}
\lstdefinestyle{hs}{
  language=Haskell,
}
\lstdefinestyle{sql}{
  language=SQL,
}
\lstdefinestyle{bash}{
  language=bash,
}
\lstdefinestyle{pseudo}{
  language = Python,
  mathescape = true,
  morekeywords = {do, procedure, end, while, if, else},
}
\lstdefinestyle{ruby}{
  language=Ruby,
}
\lstdefinestyle{java}{
	language=Java,
    morekeywords={typeof, new, true, false, function, return, null, catch, switch, var, if, in, while, do, else, case, break, class, export, boolean, throw, implements, extends, import, this, int, Integer, Boolean, String, Object, Float, Double, double, float, long, void},
    ndkeywords={List, E},
	extendedchars=true,% lets you use non-ASCII characters;
    % for 8-bits encodings only, does not work with UTF-8
}
\lstdefinestyle{R}{
	language=R,
	extendedchars=true,% for 8-bits encodings only,
    % does not work with UTF-8
}

% environments: abstract
\newenvironment{abstractEN}{%
\thispagestyle{empty}%
\selectlanguage{english}%
\begin{center}%
\textbf{\abstractname}
\end{center}}{}

\newenvironment{abstractDE}{%
\thispagestyle{empty}%
%\selectlanguage{ngerman}%
\begin{center}%
\textbf{\abstractname}
\end{center}}{}
 % load settings

\usepackage[latin9]{luainputenc}
\usepackage{color}
\usepackage{array}
\usepackage{float}
\usepackage{textcomp}
\usepackage{amsthm}

%%%%%%%%%%%%%%%%%%%%%%%%%%%%%% LyX specific LaTeX commands.
%% Because html converters don't know tabularnewline
\providecommand{\tabularnewline}{\\}
\floatstyle{ruled}
\newfloat{algorithm}{tbp}{loa}
\providecommand{\algorithmname}{Algorithm}
\floatname{algorithm}{\protect\algorithmname}

%%%%%%%%%%%%%%%%%%%%%%%%%%%%%% Textclass specific LaTeX commands.
\newenvironment{lyxcode}
	{\par\begin{list}{}{
		\setlength{\rightmargin}{\leftmargin}
		\setlength{\listparindent}{0pt}% needed for AMS classes
		\raggedright
		\setlength{\itemsep}{0pt}
		\setlength{\parsep}{0pt}
		\normalfont\ttfamily}%
	 \item[]}
	{\end{list}}
\theoremstyle{plain}
    \ifx\thechapter\undefined
      \newtheorem{lem}{\protect\lemmaname}
    \else
      \newtheorem{lem}{\protect\lemmaname}[chapter]
    \fi

%%%%%%%%%%%%%%%%%%%%%%%%%%%%%% User specified LaTeX commands.
\definecolor{gray}{rgb}{0.5, 0.5, 0.5}
\definecolor{keyword}{RGB}{243, 60, 114}
\definecolor{comment}{RGB}{0, 200, 101}
\definecolor{background}{rgb}{0.9, 0.9, 0.9}

\usepackage{xparse}
\usepackage{algorithm,algpseudocode}
\DeclarePairedDelimiter{\ceil}{\lceil}{\rceil}

\begin{document}

\lehead*{}
\rehead*{Chapter \upshape\thechapter: \headmark}
\cehead*{}
\lefoot*{}
\refoot*{\pagemark}
\cefoot*{}


\lohead*{Chapter \upshape\thechapter: \headmark}
\rohead*{}
\cohead*{}
\lofoot*{\pagemark}
\rofoot*{}
\cofoot*{}

\lstset{keywordstyle={\bfseries \color{keyword}},
commentstyle={\color{comment}},
caption={Search function for exact matches for a query of size 0 < m < k $^{[1]}$},
captionpos=b,
backgroundcolor={\color{background}},
basicstyle={\ttfamily\small},
language={C++},
numbers=none,
captionpos=b,
tabsize=2}
\providecommand{\lemmaname}{Lemma}
\renewcommand{\lstlistingname}{Listing}

\newcommand{\institute}{Institute of Computer Science}
\newcommand{\department}{Department of Computer Science and Mathematics}
\newcommand{\university}{Freie Universität Berlin}
\newcommand{\researchGroup}{Databases and Information Systems}
\newcommand{\city}{Berlin}
\newcommand{\country}{Germany}

%%%%%%%%% Begin Titlepage %%%%%%%%%%%
\begin{titlepage}
  \thispagestyle{empty}%
    \begin{picture}(2,0)(0,-1.35)
    \includegraphics[height=\logosize]{images/agdb_logo}
  \end{picture}\hfill
  \begin{picture}(4,0)(1.6,-1.35)
    \includegraphics[height=\logosize]{images/fu_logo}
  \end{picture}\vskip0pt
  \vspace*{-1.1cm}%
  \hskip-\dimexpr\abstand+\randI\relax%
  \colorbox{DBbluedark}{%
  \begin{minipage}[t]%[2\abstand][c]
  	{\paperwidth}%
    \vspace{2ex}
    \hskip\dimexpr\abstand+\randI\relax%
      \begin{minipage}{\dimexpr\paperwidth-\randA-\randI\relax}
      \color{white}%
      \LARGE\thetitle
      \end{minipage}
    \vspace{2ex}
  \end{minipage}%
  }\vskip-8.5pt
  \hskip-\dimexpr\abstand+\randI\relax%
  \colorbox{lightgray}{%
  \begin{minipage}[t][1cm][c]{\paperwidth}
    \hskip\dimexpr\abstand+\randI\relax%
    \textit{A \textbf{thesis} presented for the degree of}
    \textbf{\textit{\degree}}.
  \end{minipage}}\vfill%\vspace{5\baselineskip}

%%%%%%%%%%% datas %%%%%%%%%%%
    \textbf{\theauthor}, \university, \country\par
    Matriculation number: \matrikelno\par
    \href{mailto:\email}{\email}\par
    \vspace{0.5\baselineskip}
    \thedate\par
    \vfill%vspace{5\baselineskip}

    % SUPERVISOR
    \textbf{\underline{Supervisor:}}\par
    \textbf{\supervisor}%
      \footnote{%
      \supervisorDepartment,
      \supervisorAG},
      \supervisorUniversity,
      \supervisorCountry
    \vspace{1.5\baselineskip}

	% 1. ADVISOR
    \textbf{\underline{Reviewers:}}\par
    \textbf{\fstAdvisor}%
      \footnote{\fstAdvisorsDepartment,
      \fstAdvisorsAG},
    \fstAdvisorsUniversity,
      \fstAdvisorsCountry\par
    % 2. ADVISOR
    \textbf{\sndAdvisor}%
      \footnote{\sndAdvisorsDepartment,
      \sndAdvisorsAG},
      \sndAdvisorsUniversity,
      \fstAdvisorsCountry
   	\par\vspace{3.5\baselineskip}

    \textbf{\underline{Citation:}}\par
    \textbf{\theauthor},
    \textit{\thetitle}, \university,
    \thesisKind{} Thesis, \Year\par
    \vspace{2.5\baselineskip}

    \textbf{Version:} \versionnumber
    \vfill
\end{titlepage}




\chapter*{Statutory Declaration}
\thispagestyle{empty} % no page number
\setcounter{page}{1}

I declare that I have developed and written the enclosed \thesisKind{} thesis completely by myself, and have not used sources or means without declaration in the text. Any thoughts from others or literal quotations are clearly marked.\par
The \thesisKind{} thesis was not used in the same or in a similar version to achieve an academic grading or is being published elsewhere.\vspace{2\baselineskip}

Berlin (Germany), \thedate\vspace{2.5\baselineskip}

\rule{15.5em}{1pt}\newline
\theauthor

% Abstract in english
\begin{abstractEN}
As demand for large-scale genomic sequence matching rises, as does
demand for the ability to efficiently compute the positions or number of occurrences
of a nucleotide sequence of length k called a kmer. A kmer-index optimized for this
purposes was implemented and it's performance evaluated. The implementation
is capable of searching kmers of arbitrary length specified at runtime,
utilizes parallelization for faster construction and it's internal
structure can be further customized at compile time to achieve greater
performance for certain query sizes. The nature of the internal structure
and search functions are explained and their performant nature demonstrated
through benchmarks. Results indicate that for the purpose of finding
positions or number of occurrences for queries of
length $k\in\{3,4,...,30\}$ compared to seqan3s fm\_index the kmer-index
performs up to 60\% faster and is thus recommended for use. For arbitrarily
long queries the kmer-index was shown to exhibit speedup or slowdown
of $\pm5$\% depending on text length and is therefore in it's current
iteration applicable if not necessarily better suited for this purpose.
\end{abstractEN}
\vfill

% abstract in german
\begin{abstractDE}
Mit steigender Nachfrage f\"ur gro\ss fl\"achige genomisches Sequence Matching steigt
auch die Nachfrage Positionen f\"ur kurze genomische Sequenzen oder kmer performant
in einem Text zu finden. Zu diesem Zweck wurde ein sogenannter kmer-Index implementiert
und dessen Performance evaluiert. Die Implementierung ist f\"ahig kmer von willk\"urlicher
L\"ange zu finden, benutzt Parallelisierung f\"ur schneller Konstruktion und die interne
Struktur des Indices kann weiterhin personalisiert werden um bessere Performance
f\"ur bestimtte Query L\"angen zu erhalten. Die Natur der Personalisierung und die daraus 
resultierende Performance wurde mit Benchmarks demonstriert. Ergebnisse legen nah, 
dass f\"ur kmer der L\"ange $k\in\{3,4,...,30\}$ im Vergleich zu seqan3s fm\_index der 
kmer-index bis zu 60\% schneller ist. F\"ur kmer beliebiger L\"ange zeigen Ergebnisse, 
dass ein Speedup von $\pm5$\% je nach Textl\"ange erreicht wird. Der Einsatz der kmer-Index
Implementierung wird daher f\"ur kurze kmer empfohlen w\"ahrend f\"ur l\"angere kmer der Index 
je nach Text L\"ange einsetzbar aber nicht unbedingt besser geeignet ist.
\end{abstractDE}
\vfill

%%% TOC %%%
\tableofcontents{}
\thispagestyle{empty}
\setcounter{page}{0}

% 01: Introduction and motivation
%%% MOTIVATION %%%
\chapter{Introduction}
A k-mer (also known as "q-gram", henceforth written as "kmer") is any oligonucleotide of length $k$ usually in the
context of a sub-sequence of a longer genome. A kmer-index is a data structure that for every unique kmer
stores the position of all occurences in a given text, usually relative to the beginning of the text.
This allows the user to efficiently look up both how often and exactly where a kmer occurs.
Analysis of kmers is used in de novo (meta-)genome\cite{megahit} and transcriptome\cite{SOAPdenovo-Trans} assembly.
In these applications efficiently accessing short reads from a set of next-generation sequencing data as well as sorting
kmer that represent the edges of for example de bruijin graphs during assembly may present a
"[performance] bottleneck" (Li et. al) that may be somewhat alleviated with an efficient implementation of a kmer-index.
\newline
A kmer spectrum is a frequency spectrum that for a given $k$ depicts how many occurences kmer of a certain multiplicity in a text
exhibit. %how many of the kmer that have x copies in the text are there
Because the kmer index provides the number of occurences of each kmer, building the kmer-index inherently also yields
the kmer spectrum. Kmer spectra have been used along traditional biomarkers for phenotypic classification of cancer
metagenome samples via the aid of machine learning\cite{phenotype:classification:with:kmer:spectrum}, pairwise dissimilarity
analysis comparing metagenomes of arbitrary origin\cite{kmer:spectrum:dissimilarity} as well as assigning taxonomic labels
to such metagenomes\cite{kraken:metagenome:classification}. Kmer spectra have furthermore been utilized for estimating genome
size without the need of assembly\cite{genome:size:estimation} and for substitution-based error correction
of next generation illumina sequencing data\cite{musket:kmer:spectrum:error:correction}.
\newline
Given the wide applicability of efficiently searching kmer a wide variety of tools and libraries have been developed.
One such tool is \href{https://github.com/seqan/seqan3}{Seqan3}s fm\_index\cite{fm:index:master:thesis}, an implementation
of the commonly used fm-index first created by P. Ferragina and G. Manzini\cite{original:fm} that aims to provide a generalist
tool to finding sequences both exact and non-exact in arbitrary genomic data.
In order to allow for further optimization unlike the fm-index the kmer-index implementation does not currently allow for
searching of non-exact sequences and is thus more focused on specifically searching kmer. This paper aims to compare the applicability
of the kmer-index implementation with the Seqan3s fm\_index for this purpose and attempt to illuminate wether utilizing
the kmer-indices inherent strengths may results in an overall comparative performance boost.

% 02: Implementation
\chapter{Implementation}

% CONSTRUCTION
\section{Construction}
The kmer-index utilizes an unordered map as its central data structure.
For each kmer said map notes the position of all occurrences of a given kmer in the text.
To save on memory the kmers are converted to an unsigned integer via
the following hash function:
\begin{verse}
let $kmer =(q_{1},\,q_{2},\,...,q_{k})$ where $q\in A \text{the Alphabet}$

$hash(kmer) = \sum_{i=0}^{k}\:r(q_{i})\:\sigma^{k-i-1}$
\begin{verse}
where $\sigma=\#A,r(q_{i})\in\{0,1,...,\sigma-1\}$ the rank of $q_{i}$
\end{verse}
\end{verse}
This hash guarantees no collisions and is furthermore
\href{http://docs.seqan.de/seqan/3-master-user/group__views.html\#ga6e598d6a021868f704d39df73252974f}{also used} in the Seqan3
library with which the kmer-index is intended to interface.

\begin{algorithm}[H]
\begin{verse}
\textbf{Input}: text

\textbf{let} $n\leftarrow$ text.size()

\textbf{for} $i$ \textbf{in} $\{1,2,...,n\}$ \textbf{do}
\begin{verse}
\textbf{let} $h\leftarrow$ hash(text.substring($i,\:i+k$))

$\text{index}[h]\leftarrow(\text{index}[h],\:i)$
\end{verse}

\textbf{end}

\textbf{return}
\end{verse}
\caption{Construction of the kmer index.}
\end{algorithm}

Constructing the kmer-index is fairly straightforward: While keeping track of the current position we iterate through the
text one letter at a time, inserting the current position $i$ for the current kmer into the index.
As each position $i<n-k$ represents the position of a kmer, $n-k$ hashes will be generated and the same number of insertion
into the indices map will be performed. Construction therefore has linear amortized complexity. This is acceptably fast because genomic data is
usually static and as such the construction will usually only be done a single time after which the index is serialized
so it can be loaded directly at a later point.
\newpage
% SEARCHING
\section{Searching}

The kmer-index implementation is capable of searching queries of arbitrary length regardless of the $k$ specified on
construction. However a $k$ will still need to be chosen and depending on the length of the query in relation to $k$
performance may vary drastically.

% M = K
\subsection{Query Length m = k}

The optimally performing case is a query of length exactly $k$. In this case only a simple lookup in the unordered map will
return all positions of the query directly:
\begin{lstlisting}[caption={Search function for queries of size k.},language={[GNU]C++},tabsize=4]
unordered_map<size_t, std::vector<uint32_t>> _data;
uint64_t hash((...) query) const {(...)

template<std::ranges::range query_t>
std::vector<size_t>& search_k(query_t& query) const
{
	const auto* it = _data.at(hash(query));
	if (it == _data.end())
		return std::vector<size_t>();
	else
		return *it;
}
\end{lstlisting}

By nature of using an unordered map, querying it for the positions of a single entry has constant time amortized complexity
dependent on the number entries in \lstinline{_data} which is equivalent to the number of pairwise different kmers in the text. C++s
\href{https://en.cppreference.com/w/cpp/language/copy_elision}{return value optimization} ensures that the positions are
never actually copied and only a reference to them is moved between functions. This means that the only overhead of querying
the kmer-index compared to a simple lookup in an unordered map is the time it takes for the hash function
to compute the hash. This is the kmer-indices main strength and queries of length $k$ should be considered optimal.

\begin{lem}
\label{Lemma 1}
Search performance for queries of size $m:\,m=k$ is dependent on the number of pairwise different kmers in the text.
\end{lem}
\newpage
% M < K
\subsection{Query Length m < k}

To be able to search queries of arbitrary length without modifying the indices internal structure as dictated by the
initial choice of $k$, we cannot lookup the query directly.

\begin{verse}
let $query=(q_{1},q_{2},...,q_{m})$ where $m<k$ \newline
for an arbitrary $kmer=(s_{1},...,s_{k})$ it holds true that:

iff $\forall i\leq m:\:q_{i}=s_{i}$ then any position $pos$ of $kmer$
is also a position of $query$
because the $m$ characters proceeding $pos$ are
$s_{0},...,s_{m},...,s_{k}$ and $s_{0},...,s_{m}=q_{0},...,\,q_{m}=query$
\end{verse}

To find the position of a query we can thus generate all possible kmers that contain the query as a prefix and look
these up instead. Any valid occurence for any of these kmers is also a valid occurence of the query.
To save on memory and runtime we need to generate the hashes of the set of kmers directly.

\begin{verse}
let $hash(q_{1},q_{1},...,q_{m})=\sum_{i=0}^{k}\:r(q_{i})\:\sigma^{k-i}=h_{q}$
constant as given by $query$

let $H\subset\mathbb{Z}^{+}\coloneqq$ set of all hashes of kmer with
a prefix equal to$query$

let $h_{min},\:h_{max}:\:\forall h_{i}\in H:\:h_{min}\leq h_{i}<h_{max}$
be the strict lower and upper bound contained in $H$

~

to find $h_{min}$ we observe that as the query $q_{min}:\,r(q_{min})=h_{min}$
has a prefix equal to $query$ it holds true that

$hash(h_{min})\geq h_{p}$ because the first $m$ summands $r(q_{min,j})\:\sigma^{k-j-1}:1\leq j\leq m$
of the hash are given by the prefix

we choose the other summands $r(q_{i})\:\sigma^{k-i-1}:i>m$ to all
be as small as possible by choosing characters such that $\forall q_{i}:\:r(q_{i})=0$

thus $h_{min}=h_{p}+\sum_{i=m+1}^{k}\:0*\sigma^{k-i-1}=h_{p}$

~

to find $h_{max}$ we observe that $\#H=\sigma^{k-m}$ because $m$ characters
of each query are provided by the prefix

we furthermore observe that for two hashes $h_{a},\,h_{b}\in H:\,h_{a}<h_{b}$ the
difference between the hashes $h_{a}-h_{b}\geq1$

this is because given $q_{a}=(a_{1},\,...,\,a_{k-1},\,a_{k})\::hash(q_{a})=h_{a}\neq h_{max}-1$
to find the next smallest hash that is also in $H$, we replace the
last letter $a_{k}$ with $\alpha_{k}$ such that $r(a_{k})=r(\alpha_{k})+1$.
If $r(a_{k})=\sigma-1$ we instead substitute $a_{k-1}$, etc.

this means $hash(q_{a})$ increases by $(r(a_{k})\:\sigma^{k-(k-1)})-(r(\alpha_{k})\:\sigma^{k-(k-1)})=1$

Given this information we can conclude $H=\{h_{p},\,h_{p}+1,\,...,\,h_{p}+\sigma^{k-m}-1\}$
\end{verse}

\begin{minipage}{\linewidth} % prevent linebreak
\begin{lstlisting}[caption={Search function for queries of size 0 < m < k.},language={[GNU]C++},tabsize=2]
std::vector<size_t> check_last_kmer((...)) const;

template<std::ranges::range range_t>
std::vector<size_t> search_sub_k(range_t& query) const
{
	size_t h_p = hash(query);
	size_t h_min = 0 + h_p;
	size_t h_max = h_p + pow(_sigma, k-query.size());

	// lookup each hash
	std::vector<size_t> output_positions;
	for (size_t h = h_min; h < h_max; ++h)
	{
		for (size_t pos : _data.at(h))
			output_positions.push_back(pos);
	}

	// cover edge case for last kmer
	for (size_t pos : check_last_kmer(query))
		output_positions.push_back(pos);

	return output_positions;
}
\end{lstlisting}
\end{minipage}

Note that after looking up all the hashes we also need to call \lstinline{check_last_kmer}. This function covers an
edge case were the query happens to be a substring of the last kmer in the text. As the query is compared
against prefixes and there is no kmer with a position $p>n-k$ the query is instead
manually compared against the last $k-1$ letters of the text.

While searching queries of length $m<k$ is significantly more costly, it is feasible to search queries in an a
adequately fast manner if $k-m$ is sufficently small. The smaller $k-m$ the longer the prefix is compared to $k$ which
in turn means $\#H$ (and thus the number of hashes that need to be searched) is also lower increasing runtime.

\begin{lem}
\label{Lemma 2}
search performance for queries of size $m:\,(k\mod m)\neq0\:\land m<k$ is inversely proportional to $k-m$.
\end{lem}

The actual implementation throws an exception if $\sigma^{k-m}>10^{7}$ in order to avoid a badly chosen $m$ and $k$
combination to take hours to search for if the number of resulting hashes is too high.\footnote{While somewhat arbitrarily
chosen, $10^{7}$ represents the case of $k-m>11$ for the nucleotide alphabet ($\sigma=4$) which should allow most users
to be able to not encounter the exception during proper usage of the kmer-index and notably will mean for $k=10$ queries
of all length be accepted.}

% QUERY SIZE M > K
\subsection{Query Length m > k}

To search queries of length $m>k$ the query is split into parts of length $k$ which are then searched individually.
If $m\mod k\neq0$ there will be a "rest" of length $r<k$.

We observe that the set of positions for a specific query of length $m>k$ is a subset of the positions of the queries
prefix $p_{1}$ of length k. To confirm wether a specific position of $p_{1}$ is also a valid position of the query we
cross-reference the positions of the following parts as such:

\begin{verse}
let query $q=(q_{1},\,q_{1},\,...,\,q_{m})$ be of length $m:\:(m>k)\,\land(\,m\mod k\neq\text{0})$

let $p_{i}=(q_{i},\,...,q_{i+k})$ with $i\in[1,\,(m-(m\mod k))/k]\subset\mathbb{N}$
be the $i$-th $k$-long part of the query

let $r=(q_{m-(m\mod k)/k},\,...,q_{m})$ be the rest of length $(m\mod k)$

then the query occurs at positions $\rho_{seed}\in pos(q_{1},...q_{k})=pos(p_{1})$
iff

for all $i$ there exists a position $\rho_{2}\in pos(p_{2})$ such
that $\rho_{2}=\rho_{seed}+k$, and there exists a position $\rho_{3}\in pos(p_{3})$
such that $\rho_{3}=\rho_{2}+k$, etc.

For $\rho_{(m-(m\mod k))/k}\in pos(p_{last})$ we need to check for
a position $\rho_{rest}\in pos(rest)$ such that $\rho_{rest}=\rho_{(m-(m\mod k))/k}+k$

If a $\rho$ is found to exist for all parts $\rho_{seed}$ is confirmed to be a valid position of the query
\end{verse}

For performance purposes if at any point the program does not find an appropriate $\rho_{i}$ the current position
$\rho_{seed}$ is marked as invalid and the loop moves onto the next. While in the worst-case scenario (all positions
in $\rho_{seed}$ are valid) performance will be relatively costly in praxis especially for longer queries this is rarely the case
because given a randomized text of constant size a longer query is statistically likely to have fewer valid
occurences than a shorter query.

\begin{lem}
\label{Lemma 3}Search time for queries of size $m:\,m>k$ scales
proportionally with the number of occurences of all the corresponding $p_{i}$.
\end{lem}

Searching queries of length $m:m\mod k=0$ (which do not have a rest)
should be preferred because in that case every $p_{i}$ can be searched with the the well-performing \lstinline{search_k}.
If there is a rest \lstinline{search_subk} will need to be employed resulting in overall worse performance.

\section{Multi kmer-Index}

As implied by the above lemmas correct choice of k relative to the queries size is paramount to achieving acceptable
performance. In practical applications however the query length cannot be controlled and may be highly variable within
a set of queries. To mitigate this the kmer-index was extended to unify indices for multiple k
into one "multi" kmer-index. Given a query the multi kmer-index chooses the optimal k for searching the query available
to it in order to achieve more flexiblity and more consistent performance compared to the "single" kmer-index.

\begin{lstlisting}[caption={Class Definition and Constructor implementation of the (multi) kmer-index.},language={[GNU]C++},tabsize=2]
template<size_t k>
class single_kmer_index;

template<size_t... ks>
struct multi_kmer_index
        : public single_kmer_index<ks>...
{
	// ctor
	template<std::ranges::range text_t>
	multi_kmer_index(text_t& text)
		: single_kmer_index<ks>(text)...
	{
		(...)
	}
}

\end{lstlisting}


% 03: Methods
\chapter{Measuring Performance}

% UNORDERED MAP: METHODS
\begin{minipage}{\linewidth}
\section{Unordered Map}
\subsection{Methods}
As the unordered map is the central data structure of the kmer-index choosing the best performing available
implementation contributes tremendously to the performance of the kmer-index. As C++s \lstinline{std::unorderd_map}
did not prove sufficiently performant enough the following maps implemented by 3rd parties were tested and evaluated
specifically in the context of use within the kmer-index:

\begin{itemize}
\item \lstinline{std::unordered_map} from the \href{https://en.cppreference.com/w/cpp/container/unordered_map}{C++17 standard library}
\item \lstinline{boost::unordered_map} from the \href{https://www.boost.org/doc/libs/1_65_0/doc/html/boost/unordered_map.html}{1.65.1 Boost Library}
\item \lstinline{absl::node_hash_map} from Googles \href{https://abseil.io/docs/cpp/guides/container\#abslnode_hash_map-and-abslnode_hash_set}{Abseil}
\item \lstinline{robin_hood::unordered_map} as provided by \href{https://github.com/martinus/robin-hood-hashing}{Martin Ankerl et. al.}
\end{itemize}

Each map was filled with a fixed number of randomized integers simulating both the hash and position of kmer in the
final implementation and the duration to return a specific but randomized element was measured. Only retrieval was tested
here since after construction of the kmer-index no insertions will take place during runtime.

% UNORDERED MAP: RESULTS
\subsection{Results}
\begin{figure}[H] \label{}
\includegraphics[width=1\textwidth]{/home/clem/Workspace/kmer_index/source/benchmarks/map_vs_map/map_vs_map}

\caption{Search performance for different map implementations over size of
map.}
\end{figure}
\end{minipage}

Results indicate that \lstinline{robin_hood::unordered_map} performed
best regardless of the number of elements contained and was as such
used as the central data structure for the kmer-index.

% SINGLE KMER: METHODS
\section{Single kmer-Index}
\subsection{Methods}
To further investigate the behavior implied by Lemmas 1, 2 and 3 a benchmark was conducted.
Using a single kmer-index for $k=10$, randomly generated queries of length $\{6,7,...,50\}$ were searched in a
randomly generated text and the search calls duration to return was measured.

% SINGLE KMER: RESULTS
\subsection{Results}
\begin{figure}[H]
\includegraphics[width=1\textwidth]{/home/clem/Workspace/kmer_index/source/benchmarks/multi_vs_single/single_only}

\caption{\label{figure 1}Average duration to get all occurences of a query
of given length using a kmer-index for $k=\{10\}$. The inset graph
shows a``zoomed in'' view of the area of query lengths $[16,29]$.
The dotted lines mark query lengths that are a multiples of $k$.}

\end{figure}

The results show a predictable pattern substantiating the above mentioned
lemmas. Best performance is only achieved for query sizes $m:\:m\mod k=0$
as this avoids having to search a rest with \lstinline{search_subk}.
While queries of length $(n*k)-1,\,(n*k)-2:\:n\in\{1,2,...\}$ still
show acceptable performance, as the absolute difference between $m$
and $k$ increases the performance becomes worse (in accordance with
Lemma \ref{Lemma 2}). For queries of length $m=(n*k)+1$ results suggest
a runtime increase of up to $6{{}^7}\approx280\,000$ times compared
to best-case performance. Note that the peaks at $(n*k)+1$ reduce
in severity as query length increases. This is because with an increase
in query length given a text of fixed size the number of results per
query decreases and as stated in Lemma \ref{Lemma 3} performance will
improve slightly.

% MULTI KMER: METHODS
\section{Multi kmer-Index}
\subsection{Methods}
Given this behavior of the single kmer-index if we want to choose more $k$ to improve average
performance the most obvious addition would be to cover the previous $k$s
worst case performance: for example for $k_{1}=10$ we would additionally
choose $k_{2}=k_{1}+1=11$, $k_{3}=k_{2}+1$ and so on. Ideally we would
just use every possible $k$ in one multi kmer-index however due to memory limitations
this is sometimes not feasible. As observed above, runtime for query length $m=k-1$
was still satisfactory and thus $k_{i}=\{5,7,9,...,27,29,31\}$ was
chosen as a substitute to a multi kmer-index with perfect coverage. It was again constructed
over a randomized text and the search functions time to return all occurences of a randomized kmer with
fixed length was measured and compared to the results for the single kmer-index above.

% MULTI KMER: RESULTS
\subsection{Results}
\begin{figure}[H]
\includegraphics[width=1\textwidth]{/home/clem/Workspace/kmer_index/source/benchmarks/multi_vs_single/multi_vs_single}
\label{single_vs_multi}
\caption{Average search performance of multi and single kmer-index to find
all occurrences of a query of given length in a text of size $10^{6}$.}
\end{figure}

Despite the suboptimal coverage of the multi kmer-index performance is still very close to optimal. Note that at query
length 10 the single kmer-index shows faster performance than the multi kmer-index (c.f. red vertical line in figure 2.3).
This illustrates the slight difference in runtime between $k=m$ (which is the case for the single kmer-index
at $m=10$) and $k=m+1$ (which is true for the multi kmer-index) as it was provided $k=11$).
Note further how after $m=33$ runtime increases significantly. To explain this we need to investigate how the
multi kmer-index handles queries that cannot optimally be search using only a single k more closely.

% MULTI KMER: HOW TO CHOOSE K
\subsection{How to choose the appropriate k}
As demonstrated above the success of the multi kmer-inde to perform well relies heavily on both being provided the
approriate $k$ and choosing which $k$s to use for a certain query length. The latter is provided
by a table that computes for every possible query length $m$ a set of$\{k_{a},k_{b},...\}$
such that $k_{a}+k_{b}+...=m$. To achieve best performance we want
the minimal set (in regards to cardinality) of $k_{i}$ and for each of the $k_{i}$ to be chosen as big as possible which
in accordance with Lemma\ref{Lemma 3} reduces runtime for longer queries. This table is constructed with a simple
dynamic programming approach where to find the set of $k_{i}$ for a given query $m_{j}$ the already computed
set for a queries $m_{a},m_{b}<m_{j}:m_{a}+m_{b}=m_{j}$ are unified to create the new set.

\begin{table}[H]
\centering{}\caption{Set of $k_{i}$ for certain query lengths chosen from $k\in\{9,11,13,17\}$
by the multi kmer-index.}
\begin{tabular}{cc}
\toprule
query length & summands\tabularnewline
\midrule
\midrule
29 & 9+9+11\tabularnewline
\midrule
30 & 13+17\tabularnewline
\midrule
31 & 9+9+13\tabularnewline
\midrule
32 & n/a\tabularnewline
\midrule
33 & 9+11+13\tabularnewline
\bottomrule
\end{tabular}
\end{table}

Depending on the set of $k_{i}$ supplied at construction, not all query lengths may
be possible to be represented with only the supplied summands. In this
case the kmer index falls back on calling \lstinline{search_subk}
for one of the summands. To find a set of$k_{i}$ that covers most
of the queries supplied by the user, the following recommendations
should be kept in mind:

\begin{itemize}
\item no $k_{i}$ should be a multiple of another such that $\lnot\exists k_{j}:k_{i}\mod k_{j}=0$.
This is to avoid redundancy as the higher $k_{i}$will always be chosen
\item all $k_{i}$ should be as big as possible as stated above.
\item small $k_{i}$ should be avoided completey. If $k_{1}=3$ is provided
and is used for a sum, the index will have to crossreference all results
for a single 3mer which depending on the texts length may be far slower
than just searching that query with \lstinline{search_subk}. Initial
results suggest to not use any $k_{i}<10$. While they can still be supplied
the index will not use them for searching queries of bigger lengths unless unavoidable.
\end{itemize}
The author recommends for the set of $k_{i}$ to be a subset of \{9,
11, 13, 17, 19, 21, 23, 27, 29, 31\}, primes are preferable because
most queries will be able to be factorized into a subset of them.
For example $\{11,13,17,19,21\}$ will cover every query length $31<m\leq10000$
and the $k$ are small enough to be stored on less capable machines
even with bigger texts such as an entire genome. Note that if searching
for occurences of queries of length $m\in[3,31]$ the query length
should be directly supplied to the set of $k_{i}$ as a length smaller
than $min(ks)$ cannot be represented as a sum of$k_{i}$s.

Given this information we can now explain the harsh increase at $m=35$ in \ref{single_vs_multi}: For $m=33$ the index
divides the query into 3 search calls with $k=11$. For $m=34$ it will use $k=15, 19$ which both perform adequately.
For $m=35$ no such sum exists with the given $k$s so the index has to fall back to using the slower \lstinline{search_subk}
which increases runtime significantly.

\section{Comparison with fm_index}
\subsection{Methods}
To put the absolute runtime durations of the above figures into perspective a comparative analysis
of seqan3s \lstinline{fm_index}\footnote{Note that for exact string matching there is is no relevant performance
difference between the fm- and bi-fm-index.} and the multi kmer-index runtime was conducted. Both indices were
constructed over the same randomized text and queried with the same set of randomized queries. The measured runtime was
converted into relative speedup as such:

\begin{verse}
let $t_{fm<i>},t_{kmer<i>}$ the search runtime for searching a query
of length$i$ with the corresponding index

then $\text{{speedup}}(i)=\begin{cases}
+(1-(t_{kmer<i>}/t_{fm<i>})) & \text{{if}\;}t_{kmer<i>}>t_{fm<i>}\\
-(1-(t_{fm<i>}/t_{kmer<i>})) & \text{{if}\;}t_{kmer<i>}<t_{fm<i>}\\
\,0 & \text{{else}}
\end{cases}$|
\footnote{To clarify, if $\text{speedup}(a,b)=+75\%$ the $b$ has a runtime of 1.75 the runtime of $a$, that is $a$ is 75\% faster than $b$.}
\end{verse}

\subsection{Results}
\begin{figure}[H]
\includegraphics[width=1\textwidth]{/home/clem/Workspace/kmer_index/source/benchmarks/just_k/relative_speedup}

\caption{Relative speedup (in \%) for search calls of the kmer-index vs. the
fm-index per query length with a text size of $10{{}^8}$. The inset
plot shows the absolute runtime (in ns). The multi kmer-index used $k\in$
\{3, 4, ..., 9, 10, 11, 12, ..., 31\}}
\end{figure}

Results indicate that for bigger texts the kmer index will only perform
better for relatively small $k$. This is because the indices map is
limited to a size of $\sigma^{k}$ and as the kmer indices runtime
for searching a single kmer is indepent from the number of results
for that query, for bigger text higher $k$ will vastly increase the
maps size and thus the time it takes to traverse it. For sufficiently
small texts however, all $k\in\{3,30\}$ will be faster to search and
the overall speedup for short queries like these suggests that the
kmer index should be preferred to search small kmers.

\begin{table}[H]
\noindent \raggedright{}\caption{\label{table kmer faster while}Average$speedup(t_{kmer},t_{fm})$
in the interval$[3,30]$ per text length. Results rounded to 1 decimal
digit to account for noise in benchmark results.}
\begin{tabular*}{1\textwidth}{@{\extracolsep{\fill}}>{\centering}p{0.15\textwidth}>{\raggedleft}p{0.1\textwidth}>{\raggedleft}p{0.15\textwidth}>{\raggedleft}p{0.15\textwidth}>{\centering}p{0.25\textwidth}}
\toprule
text size & mean & maximum & mininum & kmer faster while\tabularnewline
\midrule
\midrule
$10{{}^3}$ & 21\% & 64.9\% & 3.0\% & k < 31\tabularnewline
\midrule
$10{{}^4}$ & 19.3\% & 64.2\% & 2.2\% & k < 31\tabularnewline
\midrule
$10{{}^5}$ & 17.3\% & 65.7\% & 0.4\% & k < 31\tabularnewline
\midrule
$10{{}^6}$ & 16.7\% & 58.8\% & 2.0\% & k < 31\tabularnewline
\midrule
$10{{}^7}$ & 9.4\% & 54.5\% & -4.6\% & k < 22\tabularnewline
\midrule
$10{{}^8}$ & 8.2\% & 62.1\% & -11.3\% & k < 17\tabularnewline
\midrule
$10{{}^9}$ & < 7\% & 48.9\% & < -16.5\% & k < 11\tabularnewline
\bottomrule
\end{tabular*}
\end{table}

In praxis searching queries of size $m<31$ is rare, for example in
in the application of read-mapping, reads are
\href{https://www.illumina.com/science/technology/next-generation-sequencing/plan-experiments/read-length.html}{rarely shorter than $m=50$}
and in other applications query lengths may go far beyond that. Because of this the above comparison was repeated and
relative speedup measured for queries of arbitrariy long length.

\begin{figure}[H]
\includegraphics[width=1\textwidth]{/home/clem/Workspace/kmer_index/source/benchmarks/multi_kmer_vs_fm/5_1000/runtime_diff_over_text_size}\caption{\label{figure 5_100}
Graphs showing $\text{speedup}(t_{kmer},t_{fm})$ over
query lengths for different text sizes. The grey area highlights query
lengths {[}5, 30{]} which were excluded when calculating the depicted median speedup per text length.
The multi kmer-index used all $k\in\{5, 6, 7, 8, 9, 10, 11, 12,
13, 15, 17, 19, 21, 23, 25, 27, 29, 31\}$.}
\end{figure}

\section{Discussion}
Overall the multi kmer-index has proved superior for searching appropriately small kmers as detailed in Table
\ref{table kmer faster while}. While more testing will need to be done to confirm this it seems reasonable to assume
that the multi kmer-index will always perform better than seqan3s fm\_index for query of length $m<10$. Queries this small
rarely see application in praxis however depending on text size the multi kmer-index may
exhibit overall speedup compared to the fm\_index for arbitrary queries and is therefore recommend for use with text
under a certain sizes. More testing needs to be done to confirm this boundary but results presented here suggest
positive speedup is achieved for text length $n<1e+07$.





% 04: Outlook
\chapter{Outlook}

To further improve performance and make the implementations runtime generally
more consistently applicable the following additional features are proposed:

% HASH
\section{>64bit Hash}
As mentioned above the datatype of the hashes is currently\lstinline{uint64_t}.
64 bit integers were chosen because the standard C++ library does
not currently support >64 bit integers natively and seqan3s kmer hash\href{http://docs.seqan.de/seqan/3-master-user/group__views.html\#ga6e598d6a021868f704d39df73252974f}{also uses them}.
However technically the size of the integer is arbitrary and expanding it to 128 or 256bit my improve performance
by increasing the maximum k that can still be searched with the overall faster $m=k$ search. $k$ is currently
limited to $<31$ (given the smallest relevant nucleotide alphabet). Increasing this boundary
may be especially important when working with bigger alphaebts such as the complete list of such as the complete list of
\href{https://www.bioinformatics.org/sms/iupac.html}{IUPAC codes}. Furthermore in the application
of read-mapping ultra shorts reads, reads are \href{https://www.illumina.com/science/technology/next-generation-sequencing/plan-experiments/read-length.html}{often below the length of 75}
. Given a sufficiently big enough integer type entire reads could be treated as a single kmer and searched as such which
may result in a significant speedup.\footnote{Given a 128bit integer, reads of size $m<64$ could be searched on their own. Given a 256bit integer this number increases to $m<128$}
Further research will have to be done to confirm wether abstracting the hash type to use for example
integers from boosts \href{https://www.boost.org/doc/libs/1_62_0/libs/multiprecision/doc/html/boost_multiprecision/tut/ints/cpp_int.html}{multiprecision header}
would actually improve performance. The x86-64 instruction set does support 128bit integers natively\footnote{As the \href{https://www.felixcloutier.com/x86/mul}{\lstinline{mul} instruction} supports multiplication of two 64-bit unsigned integers resulting in a 128bit result}
however depending on the machine this may not always be the case.

% COMPRESSION
\section{Multi kmer-Index Compression}
As has been shown the multi kmer-index is vastly superior to the single kmer-index in terms of performance.
However as the set of supplied $k$ increases so does the memory requirement. The current implementation of the
multi-kmer index for a set of $k\in\{k_{0}, k_{1}, ...\}$ uses memory equivalent to the sum of the memory used for
single kmer-indices with $k$ equal to the $k_{i}$ respectively.  The
kmer-index from Figure\ref{figure 5_100} offered decent coverage
by choosing about every second$k$ in \{9,10,...31\} but for a text
size of 10\textsuperscript{8} already occupied about 80gb of memory. While this is not unfeasible for stronger machines,
as each map uses about $\#H*64*n*32$ (where $H$ is the set of all pairwise different
hashes, $n$ is the text size) many bytes for bigger text such
as an entire genome this means using every possible $k$ is currently not
practical.  To remedy this it could be possible to implement a way
to compress the single kmer-indices contained in the multi kmer-index.
Each index contains all positions of the text in it's map exactly
once, which means in a multi kmer-index with 5 $k$s, the individual indices
contain at least $(5-1)n*32$ many bits of redundant entries in the
form of the vectors of positions for each hash. If a version of the
kmer-index is implemented that only contains all the texts positions
once while still allowing for adequate runtime performance an all-purpose
kmer-index could be proposed that simply holds information for all
possible ks regardless of user configuration and thus achieves optimal
performance in all cases.

% HYBRID INDEX
\section{Hybrid Approach}
As detailed above the performance peaks of the kmer-index are fairly
consistently predictable. Therefore a two-pronged approach is proposed in which
for queries for which we know the kmer-index will perform poorly the
searching is instead done by a seperate fm-index. This allows this theoretical hybrid index
to have the speedup the kmer offers while also covering the inherent inconsistencies with the fm-index
which performs highly consistently if slightly worse. It is possible to calculate
which queries should be searched with which index: The fm-index should be used to
search a query of length $m$ if and only if at least one of the following is true:
\begin{itemize}
\item the text size is $>10{{}^8}$ (c.f. Table\ref{table kmer faster while})
\item there is no set of$\{k_{a},k_{b},...\}$ such that$k_{a}+k_{b}+...=m$
\end{itemize}
In all other cases preferring the multi kmer-index component of the hybrid index
may results in overall speedup however further research is needed to develop a well-tested
heuristic that substantiates these recommendations and is capable of determining a more
exact classification of which queries should be searched by which index.

% Conclusion
\chapter{Conclusion}

The current kmer-index implementation is stable (c.f. Section\ref{Addendum: Correctness}
below), reasonably well optimized and the indices performance is superior
for searching kmers of small length $m<30$ (or less than
30 for bigger texts as detailed in Table\ref{table kmer faster while}).
For this purpose it achieved a performance increase of up to 65\%
and is thus well-suited for this purpose and should be preferred to more generalist
indices like the fm-index if runtime performance is important. For
query lengths past 30 the kmer-index has been shown to have a performance
increase between 5 and 10\% for smaller texts
while for bigger texts an overall speedup of $\pm2\%$ depending
on query length was observed. With further optimization and features
the kmer-index may become the decidedly more performant index for
exact string matching purposes in all situations however in it's current
iterations it is only recommended for use with appropriately small
text sizes or in cases where the set of query lenghts can be consistently predicted.




% 05: Addendum
\newpage{}

\chapter{Addendum}
\section{\label{Addendum: Correctness}Assuring Correctness}

To assure that the kmer-index returns the correct positions a \href{https://github.com/google/googletest}{test function}
was written that repeatedly compares search results of the kmer-index and
Seqan3s fm\_index for randomized queries and texts. This asserts that the
fm\_index itself is bug-free and as both indices are meant to be used
in the same library this assertion was presumed to be reasonable.

\begin{algorithm}[H]
\begin{verse}
\textbf{Input}: seed, $n$, $m$

\textbf{while} $(n>0)$ \textbf{do}
\begin{verse}
\textbf{let} text $\leftarrow$ generate\_sequence($m$, seed)

\textbf{let} kmer $\leftarrow$ kmer\_index(text)

\textbf{let} fm $\leftarrow$ fm\_index(text)

\textbf{for }$q$ \textbf{in} $\{k-1, k, k+1, ..., 2{*}k\}$ \textbf{do}
\begin{verse}
\textbf{let} query $\leftarrow$ generate\_sequence($q$, seed)

\textbf{let} result $_{fm}$$\leftarrow$ fm.search(query)

\textbf{let} result $_{kmer}$$\leftarrow$ kmer.search(query)

\textbf{assert} (result$_{fm}$ = result$_{kmer}$)
\end{verse}

\textbf{end}

seed $\leftarrow$ seed + 1

n $\leftarrow$ n-1
\end{verse}

\textbf{end}
\end{verse}
\caption{Test function comparing kmer- and fm-index results to assure correctness.}
\end{algorithm}

Given enough time and iterations over a big enough text this function will uncover possible bugs by itself.
During random generation the text was furthermore manually
modified at the end to account for edge cases (such
as the query happening at the very end of the text as detailed in
Section \ref{section m < k}). If a discrepancy between the fm- and kmer-index was uncovered
the function would report the exact seed and query size. As all the randomness in the above
algorithm is determinstic this made it easy to reproduce possible bugs. Before each benchmark run
whose results where presented in this paper correcteness was assured with $n>10{{}^6}$ iterations
using text sizes as large as the machines memory allowed for. During these runs all tested results between
the fm- and kmer-index were confirmed to be identical.

\section{Performance Optimization}
Implementation of the kmer-index was guided at every step by benchmarking
newly implemented components and comparing their performance against
other possible implementations. In this section the most relevant
of these decisions are explained. Note that C++20 was used for all
performance relevant code.
\subsection{Paralellization}
Other than the amount of memory needed the only true disadvantage of the
multi kmer-index is the fact that the time it takes to construct it increases drastically with the
number of $k_{i}$ specified. To address this a general purpose thread pool was implemented that
allows all of the single kmer-index elements to be constructed in parallel.
\begin{minipage}{\linewidth}
\begin{lstlisting}[caption={Paralell invocation of the create function for individual single kmer-index
elements during construction of the multi kmer-index.}]
template<std::ranges::range text_t>
multi_kmer_index(text_t& text)
	: single_kmer_index<ks>()...
{
	auto pool = kmer::detail::thread_pool{(...)};
	std::vector<std::future<(...)>> futures;
	(futures.emplace_back(
		pool.execute(&single_kmer_index<ks>::create, text)), ...);

	// wait to finish
	for (auto& f : futures)
		f.get();
}
\end{lstlisting}
\end{minipage}

As the number of possible $k$ is currently restriced to at most 31, modern systems with 32 or more CPU cores are
capable of constructing a single multi kmer-index with optimal coverage with no additional runtime overhead compared
to a single kmer-index.

\subsection{Choosing the fastest Pow Implementation}
\subsubsection{Methods}
By the nature of the hash, exponentiation (henceforth referred to
as "pow" in reference to the commonly used \lstinline{std::pow})
is used every search call, sometimes multiple times. Four different versions of pow were
implemented and their performance evaluated. \newpage Note that for the purpose of the kmer-index the pow implementation
will only ever be used with positive integers and has to be able to be evaluated at compile time.
\begin{itemize}
\item \lstinline{trivial_pow(base,n)}: A trivial implementation calling \lstinline{base*base} n-many
times
\item \lstinline{recursive_pow(base,n)}: Utilizing a recursive approach,
this function calls itself recursively $n$-many times and then evaluates
each call from the inside out to return the correct result based on
whether the exponent was odd or even
\item \lstinline{bit_pow(base,n)}: Utilizes bit-operations which are generally
more well-optimized on most modern machines
\item \lstinline{switch_pow(base,n)}: Instead of using a loop, this implementation
has multiple switch cases with identical code, when the function is
called a lookup in a pre-calculated table produces the correct first
switch case to start with. The result then "falls through" the rest
of the switch cases the correct number of times. Any exponentiation
that would overflow the unsigned 64 bit integer result is immediately
caught and 0 is returned instead.
\end{itemize}
\subsubsection{Results}
\begin{figure}[H]
\includegraphics[width=1\textwidth]{/home/clem/Workspace/kmer_index/source/benchmarks/pow_vs_pow/pow_vs_pow}

\caption{Boxplot showing runtime distribution of average time to compute $x^{y}$ for different pow implementations.}
\end{figure}

Results indicate that the implementation utilizing the fall-through
switch is overall the fastest and was thus used whenever
possible.

\pagebreak{}

% Bibliography
\bibliography{references}
\bibliographystyle{ieeetr}

\end{document}