%! Author = clem
%! Date = 23.11.20

%! Author = clem
%! Date = 23.11.20
\documentclass[
			twoside,
            fontsize=12pt,
            parskip=half-,
            %DIV=calc,
            titlepage,
            abstract=true, % prints Abstractname
            %listof=totocnumbered, % LOF, LOT, LOL  nummeriert im TOC
            bibliography=totocnumbered, % Referenzen nummeriert im TOC
            headings=normal,% Überschriften verkleinern
          	]{scrreprt}% scrreprt

%%%%%%%%%%%%%%%% Pakete %%%%%%%%%%%%%%%%
\usepackage{blindtext}

\usepackage{geometry}
\usepackage{calc}
\usepackage{amsmath}	% MUSS vor fontspec geladen werden
\usepackage{mathtools}	% modifiziert amsmath
\usepackage{amssymb}	% mathematische symbole, für \ceckmarks
\usepackage{algorithm,algpseudocode}
\usepackage{fancyhdr}
\usepackage{fontspec}
\usepackage{beramono}

\usepackage{microtype}
\usepackage[english]{babel}
\usepackage{lmodern}
%\usepackage{slantsc}
\usepackage[scale]{tgheros} % kapitälchen mit helvetica (ähnlich der arial)

\usepackage{scrhack} % ersetzt float
\usepackage{enumitem}
\usepackage{etoolbox}
\usepackage{multicol}
\usepackage{listings}
\usepackage{graphicx}
\usepackage{tabularx}
\usepackage{booktabs} % To thicken table lines
\usepackage[RGB,table]{xcolor}
\usepackage{multirow}        % tabular-multirow
\usepackage{colortbl}
\usepackage{csquotes}		 % inline quotation
\usepackage{comment} 		 % comment large blocks
\usepackage[colorlinks,      % hyperlinks
            linktocpage,
            allcolors=FUblue]
            {hyperref}
\usepackage{titling}
% Kopf- und Fußzeile:
\usepackage[headsepline=true,
            footsepline=true, % plainfootsepline,
            markcase=upper,
            manualmark, % manualmark / automark,
            headsepline=0.5pt, footsepline=0.5pt
            ]{scrlayer-scrpage}
%\usepackage{lastpage} % total page number

%\usepackage{biolinum} % notwendig für small capitals (läuft nicht)
%\usepackage{tocloft} % für cfttoctitlefont, um Schriftgröße zu verringern


%%%%%%%%%%%%%% Settings %%%%%%%%%%%%%%
\setkomafont{sectioning}{\rmfamily\bfseries}% setzt Ü-Schriften in Serifen, {disposition}
\rowcolors{2}{lightgray!20}{lightgray!40} % alternierende Zeilenfarbe
\setcounter{secnumdepth}{3}
\setcounter{tocdepth}{2}
%\setcounter{lofdepth}{1}
%\setcounter{lotdepth}{2}

%%%%%%%%%%%%%% Layout %%%%%%%%%%%%%%
\setlength{\unitlength}{1cm}
\newcommand{\abstand}{\dimexpr\paperwidth/18\relax}
\newcommand{\rand}{\dimexpr3.6\abstand\relax}
\newcommand{\randI}{\dimexpr1.8\abstand\relax}
\newcommand{\randA}{\dimexpr1.8\abstand\relax}
\newcommand{\logosize}{\dimexpr1.5cm\relax}
\setlength{\columnsep}{\abstand}
\ModifyLayer[addvoffset=-3pt]{scrheadings.foot.above.line}% Linie über Fußzeile nach oben schieben
%\ModifyLayer[addvoffset=+3pt]{scrheadings.head.below.line}% Linie unter Kopfzeile nach unten schieben
\setlength{\footheight}{\dimexpr2\abstand+15ex\relax}
\geometry{
    headheight = 1cm,
    headsep = \abstand,
    footskip = \dimexpr\abstand+\footheight\relax,
	left = \randI,
    right = \randA,
	top = \dimexpr1.8\abstand+\headheight+\headsep\relax,
    bottom = \dimexpr\abstand+\footskip\relax,
    marginparwidth = 0pt,
    bindingoffset = \abstand,
    %showframe,
    %includehead, includefoot
}

%%%%%%%%%%%%%% eigene Farben %%%%%%%%%%%%%%
\let\definecolor=\xdefinecolor
\definecolor{FUgreen}{RGB}{153,204,0}
\definecolor{FUblue}{RGB}{0,51,102}
\definecolor{DBblue}{RGB}{0,102,204}
\definecolor{DBbluedark}{RGB}{0,51,102}
\definecolor{GlobalBlack}{RGB}{0,0,0}

% setzt die globale Schriftfarbe auf DBbluedark
\makeatletter
\newcommand{\globalcolor}[1]{%
  \color{#1}\global\let\default@color\current@color
}
\makeatother
\AtBeginDocument{\globalcolor{GlobalBlack}}

% färbt die Chapter- und Sectionnummern ein
\renewcommand*{\chapterformat}{%
  \textcolor{DBblue}{\thechapter}\autodot\enskip%
}
\renewcommand*{\sectionformat}{%
  \textcolor{DBblue}{\thesection}\autodot\enskip%
}
\renewcommand*{\subsectionformat}{%
  \textcolor{DBblue}{\thesubsection}\autodot\enskip%
}
\renewcommand*{\subsubsectionformat}{%
  \textcolor{DBblue}{\thesubsubsection}\autodot\enskip%
}

%%%%%%%%%%%% Kopf- und Fußzeile:
\clearpairofpagestyles% Bisherige Einstellungen löschen
%\automark[section]{chapter}
\renewcommand*\chapterpagestyle{scrheadings} % scrheadings, auch Seiten bei Chapterbeginn erhalten Kopf- und Fußzeilen
% link: http://texwelt.de/wissen/fragen/6302/wo-sind-kopfzeile-und-fuzeile-bei-kapitelstart

\renewcommand{\chaptermark}[1]{\markboth{#1}{#1}} % \markright, \markleft, \markboth


\pagestyle{scrheadings}

\lehead*{}
\rehead*{Chapter \upshape\thechapter: \headmark}
\cehead*{}
\lefoot*{}
\refoot*{\pagemark}
\cefoot*{};


\lohead*{Chapter \upshape\thechapter: \headmark}
\rohead*{}
\cohead*{}
\lofoot*{\pagemark}
\rofoot*{}
\cofoot*{}

%\lehead*{}
%\cehead*{}
%\rehead*{\upshape\thechapter, \headmark} % \thechapter, \headmark [\headmark]
%%\lefoot*{}%\pageref{LastPage}
%\cefoot*{}
%\refoot*{}%\pagemark}%\parbox{\dimexpr\linewidth-2em\relax}{\upshape \textbf{\theauthor}, \textit{\thetitle}, \Year}}

%\lohead*{}%\upshape\thechapter\xspace \headmark}
%\cohead*{}
%\rohead*{}
%\lofoot*{}%\pagemark}%\parbox{\dimexpr\linewidth-2em\relax}{\upshape \textbf{\theauthor}, \textit{\thetitle}, \Year}}
%%\cofoot*{}%center odd foot}
%\rofoot*{}%\pagemark}%\pageref{LastPage}
%\pagestyle{scrheadings}

%\KOMAoptions{onpsinit=\linespread{3}\selectfont}% Kopf- und Fußzeilen dreizeilig
%\renewcommand*\chaptermarkformat{\thechapter\autodot\enskip}% Kopfzeile ohne chapterprefix

%%%%%%%%%%%%%% Fonts %%%%%%%%%%%%%%
\setmainfont{Roboto}
\setsansfont{TeX Gyre Heros}
\setmonofont{Bitstream Vera Sans Mono}
\renewcommand\UrlFont{\rmfamily} % passt URL an

% definiert textsc neu, damit es mit Arial funktioniert
\renewcommand\textsc[1]{{\sffamily\scshape#1}}

%%%%%%%%%%%%%%% Fontsize %%%%%%%%%%%%%%
% \setkomafont{chapter}{\LARGE}
% \setkomafont{section}{\Large}
% \setkomafont{subsection}{\large}
% \setkomafont{paragraph}{\normalsize}
%\renewcommand\cfttoctitlefont{\LARGE\bfseries}

%%%%%%%%%% Referenzen %%%%%%%%%%%
%\usepackage[style=numeric, bibencoding=utf8, backend=biber, sorting=nty, maxbibnames=15]{biblatex}% für bibliographie, style=authortitle/numeric/ieee, backend=biber/bibtex/bibtex8, bibencoding=ascii/utf8
\usepackage{silence}% Filter warnings issued by package biblatex starting with "Patching footnotes failed"
\WarningFilter{biblatex}{Patching footnotes failed}

%\bibliographystyle{unsrt}	 % reorders the cited references by apperance
%\bibliographystyle{plain}
%\addbibresource{content/references.bib}
%\DeclareFieldFormat{isbn}{}
%\renewcommand{\bibname}{\thesection References}

\usepackage{xpatch}
\xpatchbibmacro{author}{\printnames{author}}{\mkbibbold{\printnames{author}}}{}{} % author bold

%%%%%%%%%%%%%% Enumitem
\setlist[enumerate]{font=\color{DBblue}\bfseries, leftmargin=*}
\setlist[enumerate,1]{label=\arabic*, ref=\alph*.}
\setlist[enumerate,2]{label*=.\arabic*, ref=\alph*}
\setlist[enumerate,3]{label*=.\arabic*, ref=\alph*}
\setlist[itemize]{font=\color{DBblue}\bfseries, leftmargin=*}
\setlist[itemize,1]{label=$\blacktriangleright$, ref=\labelitemi}
\setlist[itemize,2]{label=\small$\blacktriangleright$, ref=\labelitemii}
\setlist[itemize,3]{label=\footenotesize$\blacktriangleright$, ref=\labelitemiii}%textbullet, diamond, bullet
\setlist[description]{font=\color{DBblue}\bfseries, leftmargin=*}

%%%%%%%%%%%%%%% Tables %%%%%%%%%%%%%%
\renewcommand{\arraystretch}{1.2}
\aboverulesep=0ex
\belowrulesep=0ex

%%%%%%%%%%%%%% Chapter %%%%%%%%%%%%
% reduces the space before chapter to zero
\renewcommand*{\chapterheadstartvskip}{\vspace*{\dimexpr5pt-\topskip\relax}}

%\RedeclareSectionCommand[beforeskip=\baselineskip, afterskip=\baselineskip]{chapter}
%\RedeclareSectionCommand[beforeskip=0.75\baselineskip, afterskip=0.5\baselineskip]{section}
%\RedeclareSectionCommand[beforeskip=0.75\baselineskip, afterskip=0.5\baselineskip]{subsection}
\RedeclareSectionCommand[beforeskip=0.75\baselineskip, afterskip=0pt]{paragraph}
\let\paragraphOld\paragraph
\renewcommand{\paragraph}[1]{\paragraphOld{#1}\hspace{1ex}}

%%%%%%%%%%%%%% Syntaxhighlighting %%%%%%%%%%%%%%
\renewcommand{\lstlistingname}{Source Code}% Listing -> Source Code
\lstloadlanguages{Python, C, C++, R, HTML, Haskell, Java, SQL, PHP, SPARQL, Octave, Matlab, OCL, XML, bash, Ruby}
\lstset{
   basicstyle=\color{DBbluedark}\small\ttfamily\selectfont,	% \scriptsize the size of the fonts that are used for the code
   backgroundcolor = \color{gray!20},	% legt Farbe der Box fest
   breakatwhitespace=false,	% sets if automatic breaks should only happen at whitespace
   breaklines=true,			% sets automatic line breaking
   captionpos=b,				% sets the caption-position to bottom, t for top
   commentstyle=\color{gray}\selectfont,% comment style
   frame=single,				% adds a frame around the code
   keepspaces=true,			% keeps spaces in text, useful for keeping indentation
							% of code (possibly needs columns=flexible)
   keywordstyle=\bfseries\color{DBblue}\selectfont,% keyword style
   ndkeywordstyle=\color{DBbluedark}\bfseries\selectfont,
   numbers=left,				% where to put the line-numbers;
   							% possible values are (none, left, right)
   numberstyle=\scriptsize\color{FUblue}\selectfont,	% the style that is used for the line-numbers
   numbersep=9pt,			% how far the line-numbers are from the code
   stepnumber=1,				% nummeriert nur jede i-te Zeile
   showspaces=false,			% show spaces everywhere adding particular underscores;
							% it overrides 'showstringspaces'
   %showstringspaces=false,	% underline spaces within strings only
   showtabs=false,			% show tabs within strings adding particular underscores
   flexiblecolumns=false,
   %tabsize=1,				% the step between two line-numbers. If 1: each line will be numbered
   stringstyle=\color{red}\ttfamily\selectfont,	% string literal style
   numberblanklines=false,				% leere Zeilen werden nicht mitnummeriert
   xleftmargin=\dimexpr2em\relax,	% Abstand zum linken Layoutrand
   framexleftmargin=\dimexpr2em-3pt\relax,
   xrightmargin=\dimexpr2pt\relax,					% Abstand zum rechten Layoutrand
   framexrightmargin=\dimexpr-2pt\relax,
   aboveskip=2ex,
}

%%%%%%%%%%%%%%%% definition of language %%%%%%%%%%%%%%%%
\lstdefinestyle{html}{
   language=HTML,
}
\lstdefinestyle{c}{
  language=C,
}
\lstdefinestyle{cpp}{
  language=C++,
}
\lstdefinestyle{py}{
   language=Python,
}
\lstdefinestyle{php}{
  language=PHP,
}
\lstdefinestyle{rdf}{%oder xml
  language=SPARQL,
}
\lstdefinestyle{o}{
  language=Octave,
}
\lstdefinestyle{m}{
  language=Matlab,
}
\lstdefinestyle{ocl}{%omg
  language=OCL,
}
\lstdefinestyle{xml}{
  language=XML,
}
\lstdefinestyle{hs}{
  language=Haskell,
}
\lstdefinestyle{sql}{
  language=SQL,
}
\lstdefinestyle{bash}{
  language=bash,
}
\lstdefinestyle{pseudo}{
  language = Python,
  mathescape = true,
  morekeywords = {do, procedure, end, while, if, else},
}
\lstdefinestyle{ruby}{
  language=Ruby,
}
\lstdefinestyle{java}{
	language=Java,
    morekeywords={typeof, new, true, false, function, return, null, catch, switch, var, if, in, while, do, else, case, break, class, export, boolean, throw, implements, extends, import, this, int, Integer, Boolean, String, Object, Float, Double, double, float, long, void},
    ndkeywords={List, E},
	extendedchars=true,% lets you use non-ASCII characters;
    % for 8-bits encodings only, does not work with UTF-8
}
\lstdefinestyle{R}{
	language=R,
	extendedchars=true,% for 8-bits encodings only,
    % does not work with UTF-8
}

% environments: abstract
\newenvironment{abstractEN}{%
\thispagestyle{empty}%
\selectlanguage{english}%
\begin{center}%
\textbf{\abstractname}
\end{center}}{}

\newenvironment{abstractDE}{%
\thispagestyle{empty}%
%\selectlanguage{ngerman}%
\begin{center}%
\textbf{\abstractname}
\end{center}}{}
 % load settings

\usepackage[latin9]{luainputenc}
\usepackage{color}
\usepackage{array}
\usepackage{float}
\usepackage{textcomp}
\usepackage{amsthm}

\makeatletter

%%%%%%%%%%%%%%%%%%%%%%%%%%%%%% LyX specific LaTeX commands.
%% Because html converters don't know tabularnewline
\providecommand{\tabularnewline}{\\}
\floatstyle{ruled}
\newfloat{algorithm}{tbp}{loa}
\providecommand{\algorithmname}{Algorithm}
\floatname{algorithm}{\protect\algorithmname}

%%%%%%%%%%%%%%%%%%%%%%%%%%%%%% Textclass specific LaTeX commands.
\newenvironment{lyxcode}
	{\par\begin{list}{}{
		\setlength{\rightmargin}{\leftmargin}
		\setlength{\listparindent}{0pt}% needed for AMS classes
		\raggedright
		\setlength{\itemsep}{0pt}
		\setlength{\parsep}{0pt}
		\normalfont\ttfamily}%
	 \item[]}
	{\end{list}}
\theoremstyle{plain}
    \ifx\thechapter\undefined
      \newtheorem{lem}{\protect\lemmaname}
    \else
      \newtheorem{lem}{\protect\lemmaname}[chapter]
    \fi

%%%%%%%%%%%%%%%%%%%%%%%%%%%%%% User specified LaTeX commands.
\definecolor{gray}{rgb}{0.5, 0.5, 0.5}
\definecolor{keyword}{RGB}{243, 60, 114}
\definecolor{comment}{RGB}{0, 200, 101}
\definecolor{background}{rgb}{0.9, 0.9, 0.9}

\usepackage{xparse}
\usepackage{algorithm,algpseudocode}
\DeclarePairedDelimiter{\ceil}{\lceil}{\rceil}

\makeatother

\lstset{keywordstyle={\bfseries \color{keyword}},
commentstyle={\color{comment}},
caption={Search function for exact matches for a query of size 0 < m < k $^{[1]}$},
captionpos=b,
backgroundcolor={\color{background}},
basicstyle={\ttfamily\small},
language={C++},
numbers=none,
captionpos=b,
tabsize=2}
\providecommand{\lemmaname}{Lemma}
\renewcommand{\lstlistingname}{Listing}

\begin{document}

% thesis details
\title{"Implementation and Performance Evaluation of a k-mer Index intended for Exact String Matching"}
\newcommand{\Year}{2020}
\date{1\textsuperscript{st} December \Year}
\newcommand{\thesisKind}{Bachelor}
\newcommand{\degree}{Bachelor~of~Science}
\newcommand{\versionnumber}{1.0}

% author details
\author{Clemens~Cords}
\newcommand{\matrikelno}{4872639}
\newcommand{\email}{clemens.cords@fu-berlin.de}

% supervisor details
\newcommand{\supervisor}{TODO~TODO}
\newcommand{\supervisorUniversity}{Freie Universität Berlin}
\newcommand{\supervisorDepartment}{Dept. of Computer Science and Mathematics}
\newcommand{\supervisorAG}{TODO}
\newcommand{\supervisorCountry}{Germany}

% Advisor 1 details
\newcommand{\fstAdvisor}{TODO~TODO}
\newcommand{\fstAdvisorsUniversity}{Freie Universität Berlin}
\newcommand{\fstAdvisorsDepartment}{Dept. of Computer Science and Mathematics}
\newcommand{\fstAdvisorsAG}{TODO}
\newcommand{\fstAdvisorsCountry}{Germany}

% Advisor 2 details
\newcommand{\sndAdvisor}{TODO~TODO}
\newcommand{\sndAdvisorsUniversity}{Freie Universität Berlin} % or company
\newcommand{\sndAdvisorsDepartment}{Dept. of Computer Science and Mathematics}
\newcommand{\sndAdvisorsAG}{TODO}
\newcommand{\sndAdvisorsCountry}{Germany}

\newcommand{\institute}{Institute of Computer Science}
\newcommand{\department}{Department of Computer Science and Mathematics}
\newcommand{\university}{Freie Universität Berlin}
\newcommand{\researchGroup}{Databases and Information Systems}
\newcommand{\city}{Berlin}
\newcommand{\country}{Germany}

%%%%%%%%% Begin Titlepage %%%%%%%%%%%
\begin{titlepage}
  \thispagestyle{empty}%
    \begin{picture}(2,0)(0,-1.35)
    \includegraphics[height=\logosize]{images/agdb_logo}
  \end{picture}\hfill
  \begin{picture}(4,0)(1.6,-1.35)
    \includegraphics[height=\logosize]{images/fu_logo}
  \end{picture}\vskip0pt
  \vspace*{-1.1cm}%
  \hskip-\dimexpr\abstand+\randI\relax%
  \colorbox{DBbluedark}{%
  \begin{minipage}[t]%[2\abstand][c]
  	{\paperwidth}%
    \vspace{2ex}
    \hskip\dimexpr\abstand+\randI\relax%
      \begin{minipage}{\dimexpr\paperwidth-\randA-\randI\relax}
      \color{white}%
      \LARGE\thetitle
      \end{minipage}
    \vspace{2ex}
  \end{minipage}%
  }\vskip-8.5pt
  \hskip-\dimexpr\abstand+\randI\relax%
  \colorbox{lightgray}{%
  \begin{minipage}[t][1cm][c]{\paperwidth}
    \hskip\dimexpr\abstand+\randI\relax%
    \textit{A \textbf{thesis} presented for the degree of}
    \textbf{\textit{\degree}}.
  \end{minipage}}\vfill%\vspace{5\baselineskip}

%%%%%%%%%%% datas %%%%%%%%%%%
    \textbf{\theauthor}, \university, \country\par
    Matriculation number: \matrikelno\par
    \href{mailto:\email}{\email}\par
    \vspace{0.5\baselineskip}
    \thedate\par
    \vfill%vspace{5\baselineskip}

    % SUPERVISOR
    \textbf{\underline{Supervisor:}}\par
    \textbf{\supervisor}%
      \footnote{%
      \supervisorDepartment,
      \supervisorAG},
      \supervisorUniversity,
      \supervisorCountry
    \vspace{1.5\baselineskip}

	% 1. ADVISOR
    \textbf{\underline{Reviewers:}}\par
    \textbf{\fstAdvisor}%
      \footnote{\fstAdvisorsDepartment,
      \fstAdvisorsAG},
    \fstAdvisorsUniversity,
      \fstAdvisorsCountry\par
    % 2. ADVISOR
    \textbf{\sndAdvisor}%
      \footnote{\sndAdvisorsDepartment,
      \sndAdvisorsAG},
      \sndAdvisorsUniversity,
      \fstAdvisorsCountry
   	\par\vspace{3.5\baselineskip}

    \textbf{\underline{Citation:}}\par
    \textbf{\theauthor},
    \textit{\thetitle}, \university,
    \thesisKind{} Thesis, \Year\par
    \vspace{2.5\baselineskip}

    \textbf{Version:} \versionnumber
    \vfill
\end{titlepage}




\chapter*{Statutory Declaration}
\thispagestyle{empty} % no page number
\setcounter{page}{1}

I declare that I have developed and written the enclosed \thesisKind{} thesis completely by myself, and have not used sources or means without declaration in the text. Any thoughts from others or literal quotations are clearly marked.\par
The \thesisKind{} thesis was not used in the same or in a similar version to achieve an academic grading or is being published elsewhere.\vspace{2\baselineskip}

Berlin (Germany), \thedate\vspace{2.5\baselineskip}

\rule{15.5em}{1pt}\newline
\theauthor

% Abstract in english
\begin{abstractEN}
As demand for large-scale genomic sequence matching rises, so does
the ability to efficiently get the positions or number of occurrences
of a query of length k called a kmer. A kmer-index optimized for this
purposes was implemented and it's performance evaluated. The implementation
is capable of searching kmers of arbitrary length specified at runtime,
utilizes parallelization for faster construction and it's internal
structure can be further customized at compile time to achieve greater
performance for certain query sizes. The nature of the internal structure
and search functions are explained and their performant nature demonstrated
through benchmarks. Results indicate that for the purpose of finding
positions or number of occurrences for queries of the commonly used
length$k\in\{3,4,...,30\}$ compared to the fm-index the kmer-index
performs up to 60\% faster and is thus recommended for use. For arbitrarily
long queries the kmer-index was shown to exhibit speedup or slowdown
of$\pm5$\% depending on text length and is therefore in it's current
iteration applicable if not necessarily better suited for this purpose.
\end{abstractEN}
\vfill

% abstract in german
\begin{abstractDE}
As demand for large-scale genomic sequence matching rises, so does
the ability to efficiently get the positions or number of occurrences
of a query of length k called a kmer. A kmer-index optimized for this
purposes was implemented and it's performance evaluated. The implementation
is capable of searching kmers of arbitrary length specified at runtime,
utilizes parallelization for faster construction and it's internal
structure can be further customized at compile time to achieve greater
performance for certain query sizes. The nature of the internal structure
and search functions are explained and their performant nature demonstrated
through benchmarks. Results indicate that for the purpose of finding
positions or number of occurrences for queries of the commonly used
length$k\in\{3,4,...,30\}$ compared to the fm-index the kmer-index
performs up to 60\% faster and is thus recommended for use. For arbitrarily
long queries the kmer-index was shown to exhibit speedup or slowdown
of$\pm5$\% depending on text length and is therefore in it's current
iteration applicable if not necessarily better suited for this purpose.
\end{abstractDE}
\vfill

\thispagestyle{empty}

\tableofcontents{}

\pagebreak{}

\setcounter{page}{1}

\section{Motivation}

A k-mer (also known as``q-gram'', henceforth written as``kmer'')
is any genomic sequence of length$k$, usually in the context of a
sub-sequence of a longer genome. Analysis of kmers is used in a wide
variety of contexts such as in de novo genome assembly\cite{SOAPdenovo},
taxonomic profiling\cite{phenotype:classification:with:kmer:spectrum} and sequence
classification\cite{kraken:sequence:classification}. Furthermore kmer spectra (a
graph of the multiplicity of each kmer in the set of pairwise different
kmer in the reference) have been used to assess genome similarity\cite{kmer:spectrum:dissimilarity}
and to correct errors in sequence data\cite{musket:kmer:spectrum:error:correction}.\href{https://github.com/seqan/seqan3}{Seqan3},
a newly released version of the C++ API for sequence analysis has
yet to implement a purpose-build index for kmer searching. This paper
aims to access the viability of substituting seqan3s fm-index\cite{fm:index:master:thesis}
with the presented kmer-index implementation not only for the purpose
of searching small kmers but for exact string matching in general
by comparing the indices performance to each other.

\section{Construction}

\begin{algorithm}[H]
\begin{verse}
\textbf{Input}: text

\textbf{let}$n$$\leftarrow$ text.size()

\textbf{for}$i$\textbf{in}$\{1,2,...,n\}$\textbf{do}
\begin{verse}
\textbf{let}$h$$\leftarrow$ hash(text.substring($i,\:i+k$))

index{[}$h${]}$\leftarrow$$(\text{index}[h],\:i)$
\end{verse}
\textbf{end}

\textbf{return}
\end{verse}
\caption{Construction of the kmer index.}
\end{algorithm}

The kmer-index utilizes an unordered map as it's central data structure
which for each kmer saves the position of all occurrences in the text.
To save on memory the kmers are converted to an unsigned integer via
the following hash function:
\begin{verse}
let$kmer$ =$(q_{1},\,q_{2},\,...,q_{k})$ where$q\in A$

$hash(kmer)$ =$\sum_{i=0}^{k}\:r(q_{i})\:\sigma^{k-i-1}$
\begin{verse}
where$\sigma=\#A$ ,$r(q_{i})\in\{0,1,...,\sigma-1\}$ the rank of$q_{i}$
\end{verse}
\end{verse}
This hash guarantees no hash collisions and was furthermore chosen
because the kmer-index is to be integrated and compared to the Seqan3
library which uses the same\href{http://docs.seqan.de/seqan/3-master-user/group__views.html\#ga6e598d6a021868f704d39df73252974f}{hash function}.
Constructing the kmer index is fairly straight-forward: We can simply
iterate through the text while keeping track of the current position
and adding it to the end of the appropriate entry for the current
kmer. Given a text of length$n$,$n-k$ hashes will be generated and
inserted into the index. Construction therefore has linear amortized
complexity which is acceptable because for genomic data index construction
will usually be done only once after which the index is serialized
and can then be loaded directly if needed at a later point.

\section{Searching}

To achieve greater flexibility the kmer-index implementation is capable
of searching queries of arbitrary length regardless of the$k$ that
is used to create it. While a$k$ still needs to be specified at compile
time, at runtime any query can be searched; however the search functions
performance will vary drastically depended on the queries length in
relation to$k$.

\subsection{\label{section 3.1}Query Size m = k}

The best performing cases are queries of length exactly$k$. For these
a simple lookup in the indices unordered map will return all results
at once:\footnote{All code examples henceforth are edited for brevity and only meant
for the purpose of demonstrating the programs behavior unless otherwise
specified. The actual implementations makes many concessions for the
sake of performance and would be too lengthy to print here.}
\begin{lyxcode}
\begin{lstlisting}[caption={Search function for queries of size k.},language={[GNU]C++},tabsize=4]
unordered_map<size_t, std::vector<uint32_t>> _data;
uint64_t hash((...) query) const {(...)};

template<std::ranges::range query_t>
std::vector<size_t>& search_k(query_t& query) const
{
	const auto* it = _data.at(hash(query));
	if (it == _data.end())
		return std::vector<size_t>();
	else
		return *it;
}
\end{lstlisting}

\end{lyxcode}
By nature of using an unordered map, querying a single entry has constant-time
amortized complexity dependent on the number of pairwise different
kmers in the text and therefore the number of different entries in\lstinline{_data}.
C++s\href{https://en.cppreference.com/w/cpp/language/copy_elision}{return value optimization}
ensures that the positions are never actually copied and only a reference
to them is moved between functions which other than the time it takes
for the hash-function to return,\lstinline{search_k}s runtime is
very close to that of a single lookup in the map. This is the kmer-indices
main strength and queries of length$k$ should be considered a best-case
scenario.
\begin{lem}
\label{Lemma 1}Search performance for queries of size$m:\,m=k$ is
dependent on the absolute number of pairwise different hashes in the
text
\end{lem}

\subsection{\label{section m < k}Query Size m < k}

To be able to search queries of arbitrary length without modifying
the indices internal structure for queries of length$m<k$ we need
to apply a different approach: As we cannot get the results for the
query directly we instead return the positions of all kmers that have
the query as a prefix.
\begin{verse}
let$query=(q_{1},q_{2},...,q_{m})$ where$m<k$

for an arbitrary$kmer=(s_{1},...,s_{k})$ it holds true that:

iff$\forall i\leq m:\:q_{i}=s_{i}$ then any position$pos$ of$kmer$
is also a position of$query$

because the$m$ characters proceeding$pos$ are

$s_{0},...,s_{m},...,s_{k}$ and$s_{0},...,s_{m}=q_{0},...,\,q_{m}=query$
\end{verse}
\noindent To avoid generating all kmers that contain$query$ as a
prefix and then hashing them we instead generate the hashes directly.
\begin{verse}
let$hash(q_{1},q_{1},...,q_{m})=\sum_{i=0}^{k}\:r(q_{i})\:\sigma^{k-i}=h_{q}$
constant as given by$query$

let$H\subset\mathbb{Z}^{+}\coloneqq$ set of all hashes of kmer with
a prefix equal to$query$

let$h_{min},\:h_{max}:\:\forall h_{i}\in H:\:h_{min}\leq h_{i}<h_{max}$
be the lower and upper bound of$H$

~

to find$h_{min}$ we observe that as the query$q_{min}:\,r(q_{min})=h_{min}$
has a prefix equal to$query$ it holds true that

$hash(h_{min})\geq h_{p}$ because the first$m$ summands$r(q_{min,j})\:\sigma^{k-j-1}:1\leq j\leq m$
of the hash are given by the prefix

we choose the other summands$r(q_{i})\:\sigma^{k-i-1}:i>m$ to all
be as small as possible by choosing characters such that$\forall q_{i}:\:r(q_{i})=0$

thus$h_{min}=h_{p}+\sum_{i=m+1}^{k}\:0*\sigma^{k-i-1}=h_{p}$

~

to find$h_{max}$ we observe that$\#H=\sigma^{k-m}$ because$m$ characters
of each query are given by the prefix

we furthermore observe that for two hashes$h_{a},\,h_{b}\in H:\,h_{a}<h_{b}$the
difference between the hashes$h_{a}-h_{b}\geq1$

this is because given$q_{a}=(a_{1},\,...,\,a_{k-1},\,a_{k})\::hash(q_{a})=h_{a}\neq h_{max}-1$
to find the next smallest hash that is also in$H$, we replace the
last letter$a_{k}$ with$\alpha_{k}$ such that$r(a_{k})=r(\alpha_{k})+1$.
If$r(a_{k})=\sigma-1$ we instead substitute$a_{k-1}$, etc.

this means$hash(q_{a})$ increases by$(r(a_{k})\:\sigma^{k-(k-1)})-(r(\alpha_{k})\:\sigma^{k-(k-1)})=1$

Given this information we can conclude$H=\{h_{p},\,h_{p}+1,\,...,\,h_{p}+\sigma^{k-m}-1\}$
\end{verse}
This gives us far less costly way to generate all hashes in a simple
for-loop:

\begin{lstlisting}[caption={Search function for queries of size 0 < m < k},language={[GNU]C++},tabsize=2]
std::vector<size_t> check_last_kmer((...)) const;

template<std::ranges::range range_t>
std::vector<size_t> search_sub_k(range_t& query) const
{
	size_t h_p = hash(query);
	size_t h_min = 0 + h_p;
	size_t h_max = h_p + pow(_sigma, k-query.size());

	// lookup each hash
	std::vector<size_t> output_positions;
	for (size_t h = h_min; h < h_max; ++h)
	{
		for (size_t pos : _data.at(h))
			output_positions.push_back(pos);
	}

	// cover edge case for last kmer
	for (size_t pos : check_last_kmer(query))
		output_positions.push_back(pos);

	return output_positions;
}
\end{lstlisting}

Note that after looking up all hashes depended on the query we also
need to call\lstinline{check_last_kmer}. This is to cover an edge
case were the query happens to be a sub-string of the last kmer in
the text. As the query is compared against prefixes of all kmers and
there is no kmer with a position$p>text.size()-k$ the query is instead
manually compared against the last$k-1$ letters of the text.

While searching queries of length$m<k$ is more costly than just a
simple lookup it is feasible to search queries in a adequately fast
manner if$k-m$ is sufficiently small as this means the given prefix
is in turn longer which means$\#H$ and thus the number of hashes
that need to be searched is minimal.
\begin{lem}
\label{Lemma 2}search performance for queries of size$m:\,(k\mod m)\neq0\:\land m<k$
is inversely proportional to$k-m$
\end{lem}
The actual implementation throws an exception if$\sigma^{k-m}>10^{7}$
in order to avoid a badly chosen$m$ and$k$ combination taking hours
to complete given the amount of hashes that would have to be looked
up.\footnote{While somewhat arbitrarily chosen,$10^{7}$represents the case of$k-m>11$
for the nucleotide alphabet ($\sigma=4$) which should allow most
users to be able to not encounter the exception during proper usage
of the kmer-index and notably will mean for$k=10$ queries of all
length be accepted.}

\subsection{\label{section 3.3}Query Size m > k}

To search queries of length$m>k$ the query is split into parts of
length$k$ which will then be searched individually. If$m\mod k\neq0$
there will also be a part at the end (``rest'') with a length smaller
than$k$.

We observe that the set of positions for a specific query of length$m>k$
is a subset of the positions of the first kmer$p_{1}$, the queries
prefix of length$k$. To confirm whether a specific position of$p_{1}$
is valid we cross-reference the positions of the following parts as
such:
\begin{verse}
let query$q=(q_{1},\,q_{1},\,...,\,q_{m})$ be of length$m:\:(m>k)\,\land(\,m\mod k\neq\text{0})$

let$p_{i}=(q_{i},\,...,q_{i+k})$ with$i\in[1,\,(m-(m\mod k))/k]\subset\mathbb{N}$
be the$i$-th$k$-long part of the query

let$r=(q_{m-(m\mod k)/k},\,...,q_{m})$ be the rest of length$(m\mod k)$

then the query occurs at positions$\rho_{seed}\in pos(q_{1},...q_{k})=pos(p_{1})$
iff

for all$i$ there exists a position$\rho_{2}\in pos(p_{2})$ such
that$\rho_{2}=\rho_{seed}+k$, and there exists a position$\rho_{3}\in pos(p_{3})$
such that$\rho_{3}=\rho_{2}+k$, etc.

For$\rho_{(m-(m\mod k))/k}\in pos(p_{last})$ we need to check for
a position$\rho_{rest}\in pos(rest)$ so that$\rho_{rest}=\rho_{(m-(m\mod k))/k}+k$

If a$\rho$ is found for all parts of the query is confirmed to occur
at$\rho_{seed}$
\end{verse}
For performance purposes if at any point the program does not find
a fitting$\rho_{i}$, the current position$\rho_{seed}$ is marked
as invalid and the loop moves onto the next. While in the worst-case
(all position in$\rho_{seed}$ are valid) performance would be relatively
costly, in praxis this is rarely the case, especially for longer queries.
We observe:
\begin{lem}
\label{Lemma 3}Search time for queries of size$m:\,m>k$ scales inversely
proportional to the number of results for the corresponding$p_{i}$.
\end{lem}
Searching queries of length$m:m\mod k=0$ (which do not have a rest)
should be preferred as to find the positions of the rest we will need
to employ\lstinline{search_subk} which can result in a significant
runtime increase, especially if the rest is short.\pagebreak{}

\section{Multi kmer-Index}

\subsection{\label{subsec: multi k overview}Overview}

To further investigate the behavior implied by Lemmas 1 to 3 a benchmark
was conducted. Queries of length$\{6,7,...,50\}$ were searched by
a kmer-index with a single$k=10$ and the search calls duration to
return was measured:

\begin{figure}[H]
\includegraphics[width=1\textwidth]{/home/clem/Workspace/kmer_index/source/benchmarks/multi_vs_single/single_only}

\caption{\label{figure 1}Average duration to get all occurences of a query
of given length using a kmer-index for$k=\{10\}$. The inset graph
shows a``zoomed in'' view of the area of query lengths$[16,29]$.
The dotted lines mark query lengths that are a multiples of$k$.}

\end{figure}

The results show a predictable pattern substantiating the above mentioned
lemmas. Best performance is only achieved for query sizes$m:\:m\mod k=0$
as this avoids having to search a rest with\lstinline{search_subk}.
While queries of length$(n*k)-1,\,(n*k)-2:\:n\in\{1,2,...\}$ still
show acceptable performance, as the absolute difference between$m$
and$k$ increases the performance becomes worse (in accordance with
Lemma\ref{Lemma 2}). For queries of length$m=(n*k)+1$ results suggest
a runtime increase of up to$6{{}^7}\approx280\,000$ times compared
to best-case performance. Note that the peaks at$(n*k)+1$ reduce
in severity as query length increases. This is because with an increase
in query length given a text of fixed size the number of results per
query decreases and as stated in Lemma\ref{Lemma 3} performance will
improve slightly.

Given this behavior if we want to choose more$k$ to improve average
performance the most obvious addition would be to cover the previous$k$s
worst case performance: for example for$k_{1}=10$ we would additionally
choose$k_{2}=k_{1}+1=11$,$k_{3}=k_{2}+1$ and so on. Ideally we would
just use every possible$k$ in one index however due to memory limitations
this is sometimes not feasible. As observed above, runtime for$m=k-1$
was still satisfactory and thus$k_{i}=\{5,7,9,...,27,29,31\}$ was
chosen as a substitute to a multi kmer-index with perfect coverage:

\begin{figure}[H]
\includegraphics[width=1\textwidth]{/home/clem/Workspace/kmer_index/source/benchmarks/multi_vs_single/multi_vs_single}

\caption{Average search performance of multi and single kmer-index to find
all occurrences of a query of given length in a text of size$10^{6}$.}
\end{figure}

Despite the sub-optimal coverage of the multi kmer-index, performance
is still very close to optimal. Note how at query length 10 the single
kmer-index shows faster performance than the multi kmer-index (c.f.
the red vertical line in Figure 2). This illustrates the slight difference
in performance between$k=m$ (which is true for the single kmer-index
for$m=10$) and$k=m+1$ (which is true for the multi kmer-index).
Note further how after$m=33$ (which results in 3 search calls with$k=11$
and no rest and thus still performs adequately) runtime increases
significantly. To investigate the multi-kmers index behavior for queries
that cannot be searched as a single kmer we first need to detail how
longer queries are handled outside of a single-k context.

\subsection{\label{section 4.2}How to choose the appropriate k}

The success of the multi kmer-index relies heavily on both being provided
a well-chosen set$ks$ of$k_{i}$ and furthermore using the appropriate$k_{i}$
for a given query length. The latter is achieved by a table that at
construction computes for every possible query length$m$ a set of$\{k_{a},k_{b},...\}$
such that$k_{a}+k_{b}+...=m$. To achieve best performance we want
the minimal set (in regards to cardinality) of$k_{i}$ for and we
furthermore want the$k_{i}$ are chosen as big as possible which accordant
with Lemma\ref{Lemma 3} reduces runtime for longer queries. The table
is constructed with a simple dynamic programming approach where to
find the set of$k_{i}$ for a given query$m_{j}$ the already computed
set for a queries$m_{a},m_{b}<m_{j}:m_{a}+m_{b}=m_{j}$ are referenced
back to.

\begin{table}[H]
\centering{}\caption{Set of$k_{i}$ for certain query lengths chosen from$k\in\{9,11,13,17\}$
by the multi kmer-index.}
\begin{tabular}{cc}
\toprule
query length & summands\tabularnewline
\midrule
\midrule
29 & 9+9+11\tabularnewline
\midrule
30 & 13+17\tabularnewline
\midrule
31 & 9+9+13\tabularnewline
\midrule
32 & n/a\tabularnewline
\midrule
33 & 9+11+13\tabularnewline
\bottomrule
\end{tabular}
\end{table}

Depending on the set of$k_{i}$ supplied, not all query lengths may
be possible to be represented with the supplied summands. In this
case the kmer index falls back on calling\lstinline{search_subk}
for one of the summands. To find a set of$k_{i}$ that covers most
of the queries supplied by the user, the following recommendations
should be kept in mind:
\begin{itemize}
\item no$k_{i}$ should be a multiple of another:$\lnot\exists k_{j}:k_{i}\mod k_{j}=0$.
This is to avoid redundancy as the higher$k_{i}$will always be chosen
\item all$k_{i}$ should be as big as possible. As stated above this will
decrease runtime however depending on memory limitations it may not
always be possible
\item small$k_{i}$ should be avoided completey. If$k_{1}=3$ is provided
and is used for a sum, the index will have to crossreference all results
for a single 3mer which depending on the text length may be far slower
than just searching that query with\lstinline{search_subk}. Initial
results suggest to not use any$k_{i}<10$ and while they can still
be supplied, the index will not use them for searching queries of
bigger lengths unless unavoidable.
\end{itemize}
The author recommends for the set of$k_{i}$ to be a subset of\{9,
11, 13, 17, 19, 21, 23, 27, 29, 31\}, primes are preferable because
most queries will be able to be factorized into a subset of them.
For example$\{11,13,17,19,21\}$ will cover every query length$31<m\leq10000$
and the$k$ are small enough to be stored on less capable machines
even with bigger texts such as an entire genome. Note that if searching
for occurences of queries of length$m\in[3,31]$ the query length
should be directly supplied to the set of$k_{i}$ as a length smaller
than$min(ks)$ cannot be represented as a sum of$k_{i}$s.

\subsection{Comparison with fm-Index}

To put the absolute values of the above figures into perspective,
a comparative analysis of the fm-index\footnote{Note that for exact string matching there's is no relevant performance
difference between the fm- and bi-fm-index.} and a multi kmer-index was conducted and the relative speedup measured:
\begin{verse}
let$t_{fm<i>},t_{kmer<i>}$ the search runtime for searching a query
of length$i$ with the corresponding index

then$\text{{speedup}}(i)=\begin{cases}
+(1-(t_{kmer<i>}/t_{fm<i>})) & \text{{if}\;}t_{kmer<i>}>t_{fm<i>}\\
-(1-(t_{fm<i>}/t_{kmer<i>})) & \text{{if}\;}t_{kmer<i>}<t_{fm<i>}\\
\,0 & \text{{else}}
\end{cases}$|\footnote{To clarify, if$speedup(a,b)=+75\%$ then$b$ has a runtime of 1.75
the runtime of$a$,$a$ is 75\% faster than$b$; if$speedup(x,y)=-30\%$
then$y$ has a runtime fo 0.7 the runtime of$x$,$x$ is 30\% slower
than$y$.}
\end{verse}

\subsubsection{For Query Lengths$m\in[3,30]$}

When searching for smaller queries the approriate$k$ should be directly
supplied. The relative performance for a multi kmer-index with all$k\in$
\{3,4,...,30,31\} was\href{https://github.com/google/benchmark}{benchmarked}
and compared against the fm-index:

\begin{figure}[H]
\includegraphics[width=1\textwidth]{/home/clem/Workspace/kmer_index/source/benchmarks/just_k/relative_speedup}

\caption{Relative speedup (in \%) for search calls of the kmer-index vs. the
fm-index per query length with a text size of$10{{}^8}$. The inset
plot shows the absolute runtime (in ns). The multi kmer-index use$k\in$
\{3, 4, ..., 9, 10, 11, 12, ..., 31\}}
\end{figure}

Results indicate that for bigger texts the kmer index will only perform
better for relatively small$k$. This is because the indices map is
limited to a size of$\sigma^{k}$ and as the kmer indices runtime
for searching a single kmer is indepent from the number of results
for that query, for bigger text higher$k$ will vastly increase the
maps size and thus the time it takes to traverse it. For sufficiently
small texts however, all$k\in\{3,30\}$will be faster to search and
the overall speedup for short queries like these suggests that the
kmer index should be preferred to search small kmers.

\begin{table}[H]
\noindent \raggedright{}\caption{\label{table kmer faster while}Average$speedup(t_{kmer},t_{fm})$
in the interval$[3,30]$ per text length. Results rounded to 1 decimal
digit to account for noise in benchmark results.}
\begin{tabular*}{1\textwidth}{@{\extracolsep{\fill}}>{\centering}p{0.15\textwidth}>{\raggedleft}p{0.1\textwidth}>{\raggedleft}p{0.15\textwidth}>{\raggedleft}p{0.15\textwidth}>{\centering}p{0.25\textwidth}}
\toprule
text size & mean & maximum & mininum & kmer faster while\tabularnewline
\midrule
\midrule
$10{{}^3}$ & 21\% & 64.9\% & 3.0\% & k < 31\tabularnewline
\midrule
$10{{}^4}$ & 19.3\% & 64.2\% & 2.2\% & k < 31\tabularnewline
\midrule
$10{{}^5}$ & 17.3\% & 65.7\% & 0.4\% & k < 31\tabularnewline
\midrule
$10{{}^6}$ & 16.7\% & 58.8\% & 2.0\% & k < 31\tabularnewline
\midrule
$10{{}^7}$ & 9.4\% & 54.5\% & -4.6\% & k < 22\tabularnewline
\midrule
$10{{}^8}$ & 8.2\% & 62.1\% & -11.3\% & k < 17\tabularnewline
\midrule
$10{{}^9}$ & < 7\% & 48.9\% & < -16.5\% & k < 11\tabularnewline
\bottomrule
\end{tabular*}
\end{table}


\subsubsection{For arbitrary Query Lengths}

To investigate the multi kmer-indices ability to be applied in general-purpose
exact string matching scenarios a large-scale benchmark was conducted.
For different text sizes, queries were searched with both indices
and their relative performance measured:

\begin{figure}[H]
\includegraphics[width=1\textwidth]{/home/clem/Workspace/kmer_index/source/benchmarks/multi_kmer_vs_fm/5_1000/runtime_diff_over_text_size}\caption{\label{figure 5_100}Graphs showing$speedup(t_{kmer},t_{fm})$ over
query lengths for different text sizes. The grey area highlights query
lengths {[}5, 30{]} which were excluded for the depicted median speedup.
The multi kmer-index used all$k\in$ \{5, 6, 7, 8, 9, 10, 11, 12,
13, 15, 17, 19, 21, 23, 25, 27, 29, 31\}.}
\end{figure}

The benchmark suggests that for longer queries the kmer-index provides
a slight speedup over the fm-index. With increasing text size this
speedup decreases however. This is because as mentioned in section\ref{section 3.3}
the kmer at the higher end of {[}3,30{]} take a longer time to search
for bigger texts which in turn means that the individuals kmers the
queries are made for whom the multi kmer-index prioritizes the bigger
k will take longer as well.

~

In summary, the kmer-index has proofed superior for searching appropriately
small kmers given the text length detailed in Table\ref{table kmer faster while}.
While more testing will have to be done it seems reasonable to assume
that kmer-index will always performa better for queries of length$m<10$.
For exact string matching of queries of arbitrary length the kmer
index in it's current iteration should be preferred for smaller text
sizes ($<10{{}^8}$) while the fm-index should proof more consistently
performant for larger texts. However it may be possible to improve
the kmer index with further features and if that achieves for example
an additional speedup of 5\% for longer queries, the kmer index may
perform better than the fm-index for all relevant text sizes such
as entire genomes.

\section{Outlook}

To further improve performance and make the implementation generally
more consistentlsapplicable the following additional features are
proposed:

\subsection{>64 bit Hash}

As mentioned above the datatype of the hashes is currently\lstinline{uint64_t}.
64 bit integers were chosen because the standard C++ library does
not currently support >64 bit integers natively and seqan3s kmer hash\href{http://docs.seqan.de/seqan/3-master-user/group__views.html\#ga6e598d6a021868f704d39df73252974f}{also uses them}.
However the size of the integer used for the hashes is arbitrary and
expanding it to 128 or 256 bit would improve the maximum k that can
still be searched with the overall faster$m=k$ search function as
it is currently limited to$<31$ for the smallest relevant nucleotide
alphabet. This may be especially important when working with bigger
alphabets such as the complete list of\href{https://www.bioinformatics.org/sms/iupac.html}{IUPAC codes}
for nucleotides or clear text for which$\sigma=255$. Furthermore
in the application of read-mapping, reads are\href{https://www.illumina.com/science/technology/next-generation-sequencing/plan-experiments/read-length.html}{often of relatively short length}
and being able to for example find all positions of a 50mer or 75mer
could give programs in this application a significant speedup compared
to factorizing these queries into multiple smaller kmer. Further research
will have to be done to verify wether abstracting the hash type for
the kmer-index to use for example integers of up to 1024 bit from
boosts\href{https://www.boost.org/doc/libs/1_62_0/libs/multiprecision/doc/html/boost_multiprecision/tut/ints/cpp_int.html}{multiprecision header}
could proof valuable as it may not be feasible due to memory limitation
or runtime overhead inherent to these integer types. Furthermore as$k$
and thus the number of possible hashes increases the indices maps
maximum size will also increase and as observed from Table\ref{table kmer faster while}
for larger texts bigger k do not neccessarily result in a significane
speedup.

\subsection{Multi kmer-index Compression}

As has been shown in section\ref{subsec: multi k overview} using
multiple k for the same index vastly increases the search performance
for a broad variety of queries. Ideally we would want to just choose
every possible k however memory limitations make this difficult. The
kmer-index from Figure\ref{figure 5_100} offered decent coverage
by choosing about every second$k$ in \{9,10,...31\} but for a text
size of 10\textsuperscript{8} already occupied about 80gb of memory.
While this is not unfeasible for stronger machines, as each map uses
about$\#H*64*n*32$ (where$H$ is the set of all pairwise different
hashes,$n$ is the text size) many bytes which for bigger text such
as an entire genome means using every possible$k$ is currently not
practical. To remedy this it could be possible to implement a way
to compress the single kmer-indices contained in the multi kmer-index.
Each index contains all positions of the text in it's map exactly
once which means in a multi kmer-index with 5 ks, the individual indices
contain at least$(5-1)n*32$ many bits of redundant entries in the
form of the vectors of positions for each hash. If a version of the
kmer-index is implemented that only contains all the texts positions
once while still allowing for adequate runtime performance an all-purpose
kmer-index could be proposed that simply holds information for all
possible ks regardless of user configuration and thus achieves optimal
performance in all cases.

\subsection{Hybrid Approach}

As detailed above the performance peaks of the kmer-index are fairly
consistently predictable. A two-pronged approach is proposed in which
for queries for whom we know the kmer-index will perform poorly the
searching is instead done by a desperate fm-index. This allows for
the now hybrid-index to have the speedup the kmer offers while also
covering the inherent inconsistency by instead using the fm-index
which performs highly consistently if sometimes worse. We can thus
precalculate which queries should be searched with which index. The
fm-index should be used to search a query of length$m$ if and only
if at least one of the following is true:
\begin{itemize}
\item the text size is$>10{{}^8}$ (c.f. Table\ref{table kmer faster while})
\item there is no set of$\{k_{a},k_{b},...\}$ such that$k_{a}+k_{b}+...=m$
\end{itemize}
In all other cases preferring the kmer-index component may result
in an overall speedup, however further research is needed to develop
a better tested heuristic that substantiates these recommendations
and is capable of determining a more exact text size (which may also
be dependent on the alphabet used) which when exceeded might proof
use of the hybrid index to be more trouble than it is worth given
the need for increased memory capacity.

\section{Conclusion}

The current kmer-index implementation is stable (c.f. Section\ref{Addendum: Correctness}
below), reasonably well optimized and the indices performance is superior
for searching kmers of relatively small length$m<30$ (or less than
30 for bigger texts as detailed in Table\ref{table kmer faster while}).
For this purpose it achieved a performance increase of up to 65\%
and is thus well-suited and should be preferred to more generalist
indices like the fm-index if runtime performance is important. For
query lengths past 30 the kmer-index has been shown to have a performance
increase between 5 and 10\% for smaller texts of length$m<10{{}^8}$
while for bigger texts an overall performance change of$\pm2\%$ depending
on query length was observed. With further optimization and features
the kmer-index may become the decidedly more performant index for
exact string matching purposes in all situations however in it's current
iterations it is only recommended for use with appropriately small
text sizes.

\newpage{}

\section{Addendum}

\subsection{\label{Addendum: Correctness}Assuring Correctness}

To assure that the kmer index returns the correct positions a\href{https://github.com/google/googletest}{test function}
was written that repeatedly compares results of the kmer-index and
the fm-index for randomized queries and texts. This asserts that the
fm-index itself is bug-free and as both indices are meant to be used
in the same library this assertion was presumed to be reasonable.

\begin{algorithm}[H]
\begin{verse}
\textbf{Input}: seed,$n$,$m$

\textbf{while}$(n>0)$\textbf{do}
\begin{verse}
\textbf{let} text$\leftarrow$ generate\_sequence($m$, seed)

\textbf{let} fm$\leftarrow$ kmer\_index(text)

\textbf{let} kmer$\leftarrow$ fm\_index(text)

\textbf{for}$q$\textbf{in} \{k-1, k, k+1, ..., 2{*}k\}\textbf{do}
\begin{verse}
\textbf{let} query$\leftarrow$ generate\_sequence(q, seed)

\textbf{let} result$_{fm}$$\leftarrow$ fm.search(query)

\textbf{let} result$_{kmer}$$\leftarrow$ kmer.search(query)

\textbf{assert} (result$_{fm}$ = result$_{kmer}$)
\end{verse}
\textbf{end}

seed$\leftarrow$ seed + 1

n$\leftarrow$ n-1
\end{verse}
\textbf{end}
\end{verse}
\caption{Test function comparing kmer- and fm-index results to assure correctness.}
\end{algorithm}

While given enough iterations and a big enough text size it is reasonable
to assume that this function will uncover possible bugs itself the
text was furthermore manually modified to account for edge cases (such
as the query happening at the very end of the text as detailed in
Section\ref{section m < k}). If a discrepancy between the fm- and
kmer-index was uncovered the function furthermore reported the exact
seed and query size. The randomness in\lstinline{generate_sequence}
was implemented to be deterministic which made reproducing possible
errors and what caused them reliable and easy. Before each benchmark
presented in this paper correctness was assured with$n>10{{}^6}$
iterations using text sizes$m$ as large as the machines memory allowed
for.

\subsection{Performance Optimization}

Implementation of the kmer-index was guided at every step by benchmarking
newly implemented components and comparing their performance against
other possible implementations. In this section the most relevant
of these decisions are explained. Note that C++20 was used for all
performance relevant code.

\subsubsection{Choosing the fastest Pow Implementation}

By the nature of the hash, exponentiation (henceforth referred to
as``pow'' in reference to the commonly used\lstinline{std::pow})
is used every search call sometimes multiple times. To optimize performance
four different versions of pow were implemented and their performance
evaluated. Note that for the purpose of the kmer-index the pow implementation
will only ever be used with positive integers and has\textasciicircum{}
to be able to be evaluated at compile time.
\begin{itemize}
\item \lstinline{trivial_pow(base,n)}: A trivial implementation calling$base*base$$n$-many
times
\item \lstinline{recursive_pow(base,n)}: Utilizing a recursive approach,
this function calls itself recursively$n$-many times and then evaluates
each call from the inside out to return the correct result based on
whether the exponent was odd or even
\item \lstinline{bit_pow(base,n)}: Utilizes bit-operations which are generally
more well-optimized on most modern machines
\item \lstinline{switch_pow(base,n)}: Instead of using a loop, this implementation
has multiple switch cases with identical code, when the function is
called a lookup in a pre-calculated table produces the correct first
switch case to start with. The result``falls through'' the rest
of the switch cases the correct number of times. Any exponentiation
that would overflow the unsigned 64 bit integer result is immediately
caught and 0 is returned instead.
\end{itemize}
\begin{figure}[H]
\textasciiacute \includegraphics[width=1\textwidth]{/home/clem/Workspace/kmer_index/source/benchmarks/pow_vs_pow/pow_vs_pow}

\caption{Boxplot showing runtime distribution of average time to compute$x^{y}$.}
\end{figure}

Results indicate that the implementation utilizing the fall-through
switch is overall the fastest and was thus used for the index whenever
possible.

\subsubsection{Choosing the fastest Map}

The unordered map is the data structure at the center of the kmer-index
and as such assuring fast access time even for very large map sizes
is necessary. As\lstinline{std::unordered_map} did not proof sufficiently
fast enough, other maps implemented by 3rd parties were tested and
evaluated specifically in the context of use in a kmer-index:
\begin{itemize}
\item \lstinline{std::unordered_map} from the\href{https://en.cppreference.com/w/cpp/container/unordered_map}{C++17 standard library}
\item \lstinline{boost::unordered_map} from the\href{https://www.boost.org/doc/libs/1_65_0/doc/html/boost/unordered_map.html}{1.65.1 Boost Library}
\item \lstinline{absl::node_hash_map} from Googles\href{https://abseil.io/docs/cpp/guides/container\#abslnode_hash_map-and-abslnode_hash_set}{Abseil}
\item \lstinline{robin_hood::unordered_map} as provided by\href{https://github.com/martinus/robin-hood-hashing}{Martin Ankerl et. al.}
\end{itemize}
Each map was filled with a fixed number of randomized elements and
it's time to return a specific element was benchmarked. Only retrieval
was tested here since after construction in the context of the kmer-index
no insertion will take place.

\begin{figure}[H]
\includegraphics[width=1\textwidth]{/home/clem/Workspace/kmer_index/source/benchmarks/map_vs_map/map_vs_map}

\caption{Search performance for different map implementations over size of
map.}
\end{figure}

Results indicate that\lstinline{robin_hood::unordered_map} performed
best regardless of the number of elements contained and was as such
used as the central data structure for the kmer-index.

\subsection{Parallelization}

Other than the amount of memory needed the only true disadvantage
of using the multi kmer-index is the fact that the time it takes to
construct increases drastically with the number of$k_{i}$ specific.
To address this a general purpose thread pool was implemented that
allows all of the single kmer-index elements to be constructed in
parallel:

\begin{lstlisting}[caption={Paralell invocation of the create function for individual kmer-index
elements during construction of the multi kmer-index.}]
template<std::ranges::range text_t>
multi_kmer_index(text_t& text)
	: single_kmer_index<ks>()...
{
	auto pool = kmer::detail::thread_pool{(...)};
	std::vector<std::future<(...)>> futures;
	(futures.emplace_back(
		pool.execute(&single_kmer_index<ks>::create, text)), ...);

	// wait to finish
	for (auto& f : futures)
		f.get();
}
\end{lstlisting}

As the number of possible$k$ is currently restricted to at most 31,
modern systems with 32 or more CPUs are capable of constructing one
multi kmer-index with optimal coverage with no additional runtime
overhead compared to a single-k index.

\pagebreak{}

% Bibliography
\bibliography{references}
\bibliographystyle{ieeetr}

\end{document}